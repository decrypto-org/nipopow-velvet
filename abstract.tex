\begin{abstract}
{\em Superlight} blockchain clients
learn facts about the blockchain state
while requiring merely polylogarithmic communication in the total
number of blocks.
For proof-of-work blockchains,
two known constructions exist: Superblock and FlyClient.
%
Unfortunately, none of them can be easily deployed to existing blockchains, as  they require consensus changes  and at least a soft fork to implement.

In this paper, we
investigate how a blockchain can be upgraded to support superblock clients without a soft fork. We show that it is possible to implement the needed changes without modifying the consensus protocol and by requiring only a minority of miners to upgrade, a process termed a ``velvet fork'' in the literature. While previous work conjectured that superblock clients can be safely deployed using velvet forks as-is, we show that previous constructions are insecure, and that using velvet techniques to interlink a blockchain can pose insidious security risks. We describe a novel class of attacks, called  ``chain-sewing'', which arise in the velvet fork setting: an adversary can cut-and-paste portions of various chains from independent temporary forks, sewing them together to
fool a superlight client into accepting a false claim.
We show how previous velvet fork constructions can be attacked via chain-sewing.
Next, we put forth the first provably secure velvet superblock client construction which we show  secure against adversaries that are bounded by 1/3 of the \emph{upgraded} honest miner population.
Like non-velvet superlight clients, our approach allows proving generic predicates about chains using infix proofs and as such can be adopted in practice for fast synchronization of transactions and accounts.
\end{abstract}
