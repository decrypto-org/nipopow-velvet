\section{Analysis}
In this section, we prove the security of our scheme. We use the techniques
developed in the Backbone line of work~\cite{backbone}. Towards that end, we
follow their definitions and call a round \emph{successful} if at least one
honest party made a successful random oracle query during the round, i.e., a
query $b$ such that $H(b) \leq T$. A round in which exactly one honest party
made a successful query is called \emph{uniquely successful} (the adversary
could have also made successful queries during a uniquely successful round). Let
$X_r \in \{0, 1\}$ and $Y_r \in \{0, 1\}$ denote the indicator random variables
signifying that $r$ was a successful or uniquely successful round respectively,
and let $Z_r \in \mathbb{N}$ be the random variable counting the number of
successful queries of the adversary during round $r$. For a set of consecutive
rounds $U$, we define $Y(U) = \sum_{r \in U} Y_r$ and similarly define $X$ and
$Z$. We denote $f = \Ex[X_r] < 0.3$ the probability that a round is successful.

Let $\lambda$ denote the security parameter (the output size $\kappa$ of the
random oracle is taken to be some polynomial of $\lambda$).
We make use of the following known~\cite{backbone} results.
It holds that $pqt < \dfrac{t}{n-t} \dfrac{f}{1-f}$. For the Common Prefix
parameter, it holds that $k \geq 2\lambda f$.
Additionally, for any set of
consecutive rounds $U$, it holds that
$\Ex[Z(U)] < (1 + \frac{\delta}{2})f\frac{t}{n - t}|U|$,
$\Ex[X(U)] < pq(n - t)|U|$,
$\Ex[Y(Y)] > f(1 - f)|U|$.
An execution is called \emph{typical}
if the random variables $X, Y, Z$ do not deviate significantly (more than some
error term $\epsilon < 0.3$) from their expectations. It is known that
executions are typical with overwhelming probability in
$\lambda$. Typicality ensures that for any set of consecutive
rounds $U$ with $|U| > \lambda$ it holds that
$Z(U) < \Ex[Z(U)] - \epsilon\Ex[X(U)]$ and
$Y(U) > (1 - \epsilon)\Ex[Y(U)]$.

The following definition and lemma are known~\cite{dionyziz} results and will
allow us to argue that some smooth superblocks will survive in all honestly
adopted chains. With foresight, we remark that we will take $Q$ to be the
property of a block being both smooth and having attained some superblock level
$\mu \in \mathbb{N}$.

\begin{definition}[Block property, $Q$-block]
    A \emph{block property} is a predicate $Q$ defined on a hash output $h \in \{ 0, 1 \}^\kappa$. Given a block property $Q$, a valid block with hash $h$ is called a $Q$-block if $Q(h)$ is true.
\end{definition}

\begin{lemma}[Unsuppressibility]
    Consider a collection of polynomially many block properties $\mathcal{Q}$. In a typical execution every set of consecutive rounds $U$ has a subset $S$ of uniquely successful rounds such that
    \begin{itemize}
        \item $\lvert S \rvert \geq Y(U) - 2Z(U) - 2 \lambda f (\dfrac{t}{n-t} \cdot \dfrac{1}{1-f} + \epsilon)$
        \item for any $Q \in \mathcal{Q}$, $Q$-blocks generated during $S$ follow the distribution as in an unsuppressed chain
        \item after the last round in $S$ the blocks corresponding to $S$ belong
        to the chain of any honest party.
    \end{itemize}
\end{lemma}

We now apply the above lemma to our construction. The following result lies at
the heart of our security proof and allows us to argue that an honestly adopted
chain will have a better superblock score than an adversarially generated chain.

\begin{lemma}\label{lem:claim3_lemma}
   Consider Algorithm \ref{alg:updateInterlink} under velvet fork with parameter $g$ and (1/4)-bounded velvet honest majority. Let $U$ be a set of consecutive rounds $r_1 \cdots r_2$ and $\chain$ the chain of an honest party at round $r_2$ of a typical execution. Let $\chain^S_U = \{b \in \chain: \text{b is smooth} \land \text{b was generated during U} \}$. Let $\mu, \mu' \in \mathbb{N}$.
   Let $\chain'$ be a $\mu'$ superchain containing only adversarial blocks generated during $U$ and suppose $\lvert \chain' \rvert > k$. Then it holds that
   $2^{\mu'}\lvert \chain' \rvert < 2^{\mu}\lvert \chain^S_U\upchain^{\mu} \rvert $.
\end{lemma}
\begin{proof}
  From the Unsuppressibility Lemma we have that there is a set of uniquely
  successful rounds $S \subseteq U$, such that
  $\lvert S \rvert \geq Y(U) - 2Z(U) - \delta'$, where
  $\delta' = 2 \lambda f (\dfrac{t}{n-t} \cdot \dfrac{1}{1-f} + \epsilon)$.
  We also know that $Q$-blocks generated during $S$ are distributed as in an
  unsuppressed chain. Therefore considering the property $Q$ for blocks of
  level $\mu$ that contain smooth interlinks we have that
  $\lvert \chain^S_U\upchain^{\mu} \rvert \geq (1-\epsilon)g2^{-\mu} \lvert S \rvert$.
  We also know that for the total number of $\mu'$-blocks the adversary
  generated during $U$ that
  $\lvert \chain' \rvert \leq (1+\epsilon)2^{-\mu'}Z(U)$.
  Then we have to show that
  $(1-\epsilon)g (Y(U) - 2Z(U) - \delta' ) > (1+\epsilon)Z(U)$ or
  $((1+\epsilon)+2g(1-\epsilon))Z(U) < g(1-\epsilon)(Y(U) + \delta')$.
  Let $\alpha_z = (1+\epsilon)+2g(1-\epsilon)$. Then we have to show that
  $\alpha_z Z(U) < g(1-\epsilon)(Y(U) + \delta')$.

Substituting the bounds of $X$, $Y$, $Z$ discussed above, it suffices to show that
\begin{equation*}
    \alpha_z[(1+\dfrac{\delta}{2})f \dfrac{t}{n-t} \lvert U \rvert + \epsilon pq(n-t) \lvert U \rvert ] < (1- \epsilon)g[ (1-\epsilon)f(1-f) \lvert U \rvert - \delta' ]
\end{equation*} or
\begin{equation*}
        \alpha_z f \lvert U \rvert \dfrac{t}{n-t}(1 + \dfrac{\delta}{2}) + \alpha_z \epsilon pq(n-t) \lvert U \rvert < (1- \epsilon)g[ (1-\epsilon)f(1-f) \lvert U \rvert - \delta' ]
\end{equation*} or
\begin{equation*}
        \dfrac{t}{n-t} < \dfrac{ (1- \epsilon)g[ (1-\epsilon)f(1-f) \lvert U \rvert - \delta' ] - \alpha_z \epsilon pq(n-t) \lvert U \rvert }  { f \lvert U \rvert \alpha_z (1 + \dfrac{\delta}{2})}
\end{equation*} or
\begin{equation*}
    \dfrac{t}{n-t} < \dfrac{  (1- \epsilon)g[ (1-\epsilon)f(1-f) - \dfrac{\delta'}{\lvert U \rvert} ] - \alpha_z \epsilon pq(n-t) }  { f \alpha_z (1 + \dfrac{\delta}{2})}
\end{equation*}

But
\begin{equation*}
    \dfrac{\epsilon pq(n-t)}{f(1+\dfrac{\delta}{2})} = \epsilon' \ll 1
\end{equation*} thus we have to show that
\begin{equation*}
    \dfrac{t}{n-t} < \dfrac{  (1- \epsilon)g[ (1-\epsilon)f(1-f) - \dfrac{\delta'}{\lvert U \rvert} ] }  { f \alpha_z (1 + \dfrac{\delta}{2})} - \epsilon'
\end{equation*} or
\begin{equation}\label{eq:lemma_eq2}
    \dfrac{t}{n-t} < g \cdot \dfrac{(1-\epsilon)^2 f(1-f) - \dfrac{(1-\epsilon)\delta'}{\lvert U \rvert} }{(1+\dfrac{\delta}{2})f\alpha_z} - \epsilon'
\end{equation}

In order to show Equation~\ref{eq:lemma_eq2} we use typical bound values for the parameters in our setting. We use $\delta > \dfrac{2}{3}$ since we assume a (1/4) adversary and $f \leq \dfrac{1}{10}$. We also need to estimate the value of $\dfrac{\delta'}{\lvert U \rvert}$. Because all blocks in $\chain'$ were generated during $U$ and $\lvert \chain' \rvert > k $, $\lvert U \rvert$ follows negative binomial distribution with probability $pqt$ and number of successes $k$.
Applying a Chernoff bound we have that $\lvert U \rvert > (1-\epsilon) \dfrac{k}{pqt}$. Using the inequalities $k \geq 2\lambda f$ and $pqt < \dfrac{t}{n-t} \dfrac{f}{1-f}$, we deduce that $\lvert U \rvert > (1-\epsilon)[\dfrac{2\lambda (n-t)(1-f)}{t}]$. So we have that
\begin{equation*}
    \dfrac{\delta'}{\lvert U \rvert} < \dfrac{2\lambda f (\dfrac{t}{n-t} \dfrac{1}{1-f} + \epsilon)}{(1-\epsilon)[\dfrac{2\lambda (n-t)(1-f)}{t}]}
\end{equation*} or
\begin{equation}\label{eq:lemma_eq3}
    \dfrac{\delta'}{\lvert U \rvert} < \left( \dfrac{t}{n-t} \right)^2 \dfrac{f}{(1-\epsilon)(1-f)^2} + \epsilon < 0.06 + \epsilon
\end{equation}

By substituting (\ref{eq:lemma_eq3}) and the typical parameter bounds in Equation (\ref{eq:lemma_eq2}) we conclude that it suffices to show that
\begin{equation}\label{eq:lemma_final_eq}
    \dfrac{t}{n-t} < \dfrac{1-\epsilon''}{3}g
\end{equation}

But equation (\ref{eq:lemma_final_eq}) is equivalent to $\dfrac{t}{n-t} < \dfrac{1-\delta_v}{3}g$ which is the (1/4) velvet honest majority assumption, so the claim is proven.
\end{proof}

\begin{theorem}[Suffix Proofs Security under velvet fork]
	Assuming honest majority under velvet fork conditions (\ref{defn:velvet_honest_majority}) such that $t \leq (1 - \delta_v) \dfrac{n_h}{3}$ where $n_h$ the number of upgraded honest parties, the Non-Interactive Proofs of Proof-of-Work construction for computable $k$-stable monotonic suffix-sensitive predicates under velvet fork conditions in a typical execution is secure.
\end{theorem}
\begin{proof}
By contradiction. Let $Q$ be a k-stable monotonic suffix-sensitive chain predicate. Assume for contradiction that NIPoPoWs under velvet fork on $Q$ is insecure. Then, during an execution at some round  $r_3$, $Q(\chain)$ is defined and the verifier $V$ disagrees with some honest participant. $V$ communicates with adversary $\mathcal{A}$ and honest prover $B$. The verifier receives proofs $\pi_\mathcal{A}, \pi_B$ which are of valid structure. Because $B$ is honest, $\pi_B$ is a proof constructed based on underlying blockchain $\chain_B$ (with $\pi_B \subseteq \chain_B$), which $B$ has adopted during round $r_3$ at which $\pi_B$ was generated. Consider $\widetilde{\chain}_\mathcal{A}$ the set of blocks defined as $\widetilde{C}_\mathcal{A} = \pi_\mathcal{A} \cup \{ \bigcup \{\chain_h^r\{{:}b_\mathcal{A}\}:  b_\mathcal{A} \in \pi_\mathcal{A}, \exists h,r : b_\mathcal{A} \in \chain_{h}^{r}\}  \}$ where $\chain_h^r$ the chain that the honest party $h$ has at round $r$.

The verifier outputs $\neg Q(\chain_B)$. Thus it is necessary that $\pi_\mathcal{A} {\geq}_m \pi_B$. We show that $\pi_\mathcal{A} {\geq}_m \pi_B$ is a negligible event.

Let the levels of comparison decided by the verifier be $\mu_\mathcal{A}$ and $\mu_B$ respectively. Let $b_0 = LCA(\pi_\mathcal{A}, \pi_B)$. Let $\mu'_B$ be the adequate level of proof $\pi_B$  with respect to block $b_0$. Call $\alpha_\mathcal{A} = \pi_\mathcal{A} \upchain^{\mu_\mathcal{A}}\{b_0{:}\}$, $\alpha'_B = \pi_B \upchain^{\mu'_B}\{b_0{:}\}$.

From Corollary \ref{cor:adversarial_proof_scheme} we have that the adversarial proof consists of a smooth interlink subchain followed by a thorny interlink subchain. We will refer to the smooth part of $\alpha_\mathcal{A}$ as $\alpha^{\mathcal{S}}_\mathcal{A}$ and to the thorny part as $\alpha^{\mathcal{T}}_\mathcal{A}$.

Our proof construction is based on the following intuition: we consider that $\alpha_\mathcal{A}$ consists of three distinct parts $\alpha_\mathcal{A}^1, \alpha_\mathcal{A}^2, \alpha_\mathcal{A}^3$ with the following properties.

Consider $b_0 = LCA(\pi_\mathcal{A}, \pi_B)$ the fork point between $\pi_\mathcal{A} \upchain^{\mu_\mathcal{A}}, \pi_B \upchain^{\mu_B}$ and $b_1 = LCA (\alpha^{\mathcal{S}}_\mathcal{A}, \chain_B)$ the fork point between $\pi^{\mathcal{S}}_\mathcal{A} \upchain^{\mu_\mathcal{A}}, \chain_B $ at the zero level as the honest prover could observe. Part
$\alpha_\mathcal{A}^1$ contains the blocks between $b_0$ exclusive and $b_1$ inclusive generated during the set of consecutive rounds $\mathcal{S}_1$ and $\lvert  \alpha_\mathcal{A}^1 \rvert = k_1$.
Consider $b_2$ the last block in $\alpha_\mathcal{A}$ generated by an honest party. Part $\alpha_{\mathcal{A}}^2$ contains the blocks between $b_1$ exclusive and $b_2$ inclusive generated during the set of consecutive rounds $\mathcal{S}_2$ and $\vert  \alpha_\mathcal{A}^2 \vert = k_2$. Consider $b_3$ the next block of $b_2$ in $\alpha_\mathcal{A}$. Then $\alpha_{\mathcal{A}}^3 = \alpha_\mathcal{A}[b_3{:}]$ and $\vert  \alpha_\mathcal{A}^3 \vert = k_3$ of all adversarially generated blocks generated during the set of rounds $\mathcal{S}_3$. So, $\vert \alpha_\mathcal{A} \vert = k_1 + k_2 + k_3$ and we will show that $\vert \alpha_\mathcal{A} \vert < \vert \alpha_B \vert$ .

The above are illustrated, among other, in Parts I, II of Figure \ref{fig:proof_velvet}.

\begin{figure}[h!]
	\begin{center}
    \includegraphics[width=\columnwidth]{figures/proof_velvet-crop.pdf}
	\end{center}
	\caption{Wavy lines imply one or more blocks. Dashed lines and arrows imply interlink pointers to superblocks. \textbf{I}: the three round sets in two competing proofs at different levels, \textbf{II}: the corresponding 0-level blocks implied by the two proofs, \textbf{III}: blocks participating in chain $\chain_B$ and block set $\widetilde{\chain}_\mathcal{A}$ from the verifier's perspective.}
    \label{fig:proof_velvet}
\end{figure}

We will now show three successive claims under velvet fork conditions: First that $\alpha_\mathcal{A}^1$ contains only a few blocks. Second,  $\alpha_\mathcal{A}^2$ contains only a few blocks. And third, the adversary is able to produce a winning $a_\mathcal{A}$ with negligible probability.\\

\textbf{Claim 1:} $\alpha_\mathcal{A}^1 = \alpha_\mathcal{A}(b_0 : b_1]$ contains only a few blocks. We have defined $b_0 = LCA(\pi_\mathcal{A}, \pi_B)$ and $b_1 = LCA(\alpha^{\mathcal{S}}_\mathcal{A}, \alpha'_B \downarrow)$. First observe that there are no thorny blocks in $\alpha_\mathcal{A}^1$ since $\alpha_\mathcal{A}^1[-1] = b_1$ is a smooth block. This means that if $b_1$ was generated at round $r_{b_1}$ and $\alpha^{\mathcal{S}}_\mathcal{A}[-1]$ in round $r$, then $r \geq r_{b_1}$. So, $\alpha_\mathcal{A}^1$ consists a valid chain. We show the statement considering the two possible cases for the relation of $\mu_\mathcal{A}, \mu'_B$.

\textit{\underline{Claim 1a}:} If $\mu'_B \leq \mu_A$ then they are completely disjoint. In such a case of inequality, every block in $\alpha_A$ would also be of lower level $\mu'_B$. Because of the adequate level $\mu'_B$ we know that $C\{b{:}\}\upchain^{\mu'_B} = \pi\{b{:}\}\upchain^{\mu'_B}$\cite{nipopows}. Subsequently, any block in $\pi_A\upchain^{\mu_A}\{b{:}\}[1{:}]$ would also be included in proof $\alpha'_B$, but $b=LCA(\pi_A, \pi_B)$ so there can be no succeeding block common in $\alpha_A, \alpha'_B$.

\textit{\underline{Claim 1b}:} If  $\mu'_B > \mu_A$ then $\lvert \alpha_A[1{:}]  \cap \alpha'_B\downarrow[1{:}] \rvert = k_1 \leq g(2^{\mu'_B - \mu_A})$.\\
Let's call $b$ the first block in $\alpha'_B$ after block $b_0$. Suppose for contradiction that $k_1 > g(2^{\mu'_B - \mu_A})$.  Since block $b$ of level $\mu'_B$ is also of level $\mu_A$, the adversary could include it in the proof but $b$ cannot exist in both $\alpha_A, \alpha'_B$ since $\alpha_A \cap \alpha'_B = \emptyset$ by definition. In case that the adversary chooses not to include $b$ in the proof then she can include no other blocks of $\chain_B$ in her proof, since it would not consist a valid chain. Therefore, the adversary can include at most the $\mu_\mathcal{A}$ upgraded  blocks  between  $b_0$, $b$, which are expected to be equal to $g(2^{\mu'_B - \mu_A})$.

We conclude that $\lvert \alpha^{\mathcal{S}}_\mathcal{A} \cap \alpha'_B\downarrow[1{:}] \rvert = k_{1} \leq g(2^{\mu'_B - \mu_\mathcal{A}}) $, where $g$ the velvet parameter denoting the percentage of upgraded honest parties.

Consequently, there are at least $\lvert\alpha_\mathcal{A}\rvert - k_1$ blocks after block $b$ in $\alpha_\mathcal{A}$ which are not honestly generated blocks existing in $\chain_B$. In other words, there are $\vert \alpha_\mathcal{A} \vert - k_1$ blocks after block $b$ in $\alpha_\mathcal{A}$, which are either thorny blocks existing in $\chain_B$ either don't belong in $\chain_B$.\\

\textbf{Claim 2.}
Part $\alpha_\mathcal{A}^2 = \alpha_\mathcal{A}(b_1:b_2]$ consists of only a few blocks. Let $ \lvert \alpha_\mathcal{A}^2 \rvert = k_2$. We have defined $b_2 = \alpha_\mathcal{A}^2[-1]$ to be the last block generated by an honest party in $\alpha_\mathcal{A}$. Consequently no thorny block exists in $\alpha_\mathcal{A}^2$, so all blocks in this part belong in a proper zero-level chain $\chain_\mathcal{A}^2$.  Let $r_{b_1}$ be the round at which $b_1$ was generated. Since $b_1$ is the last block in $\alpha_\mathcal{A}$ which belongs in $\chain_B$, then $\chain_\mathcal{A}^2$ is a fork chain to $\chain_B$ at some block $b'$ generated at round $r' \geq r_{b_1}$.

Let $r_2$ be the round when $b_2$ was generated by an honest party. Because an honest party has chain $\chain_B$ at later round $r_3$ when the proof $\pi_B$ is constructed and because of the Common Prefix property on parameter $k_{2\downarrow} = g \cdot 2^{\mu_\mathcal{A}} \cdot k_2$, we conclude that $k_2 \leq k$.

\textbf{Claim 3.} The adversary may submit a suffix proof such that $\lvert \alpha_\mathcal{A}\rvert \geq \lvert \alpha_B \rvert$ with negligible probability. As explained earlier part $\alpha^3_\mathcal{A}$ consists only of adversarially generated blocks. The security parameter $m$ guarantees that $\lvert \alpha^3_\mathcal{A} \rvert > k$, as will be later presented in detail. Let $U$ be the set of consecutive rounds $r_2...r_3$. Then all $k_3$ blocks of this part of the proof are generated during $U$. Let $\alpha'^{S_3}_B$ be the last part of the honest proof containing the interlinked $\mu_B$ superblocks generated during $U$. Then from lemma \ref{lem:claim3_lemma} we have that $ 2^{\mu_\mathcal{A}}\lvert \alpha^3_\mathcal{A} \rvert < 2^{\mu_B} \lvert \alpha'^U_{B}\upchain^{\mu_B} \rvert $.

From all the above Claims we have that:\\
For the first round set $S_1$, because of the common underlying chain:
\begin{equation} \label{eq_v_round_set_1}
2^{\mu_\mathcal{A}} \lvert \alpha_\mathcal{A}^{S_1} \rvert \leq 2^{\mu'_B} \lvert \alpha'{_B^{S_1}} \rvert
\end{equation}
For the second round set $S_2$ because of the adoption by an honest party of chain $\chain_B$ at a later round $r_3$ we have:
\begin{equation} \label{eq_v_round_set_2}
	2^{\mu_\mathcal{A}} \lvert \alpha_\mathcal{A}^{S_2} \rvert \leq 2^{\mu'_B} \lvert \alpha'{_B^{S_2}} \rvert
\end{equation}
For the third round set $S_3$ we have:
\begin{equation} \label{eq_v_round_set_3}
	2^{\mu_\mathcal{A}} \lvert \alpha_\mathcal{A}^{S_3} \rvert < 2^{\mu'_B} \lvert \alpha'{_B^{S_3}} \rvert
\end{equation}
Consequently we have:

\begin{equation*}
	2^{\mu_\mathcal{A}} ( \vert \alpha_\mathcal{A}^{S_1} \vert + \vert \alpha_\mathcal{A}^{S_2} \vert + \vert \alpha_\mathcal{A}^{S_3} \vert ) < 2^{\mu'_B} ( \vert \alpha'{_B^{S_1}} \vert + \vert \alpha'{_B^{S_2}} \vert + \vert \alpha'{_B^{S_3}} \vert) \Rightarrow
\end{equation*}

\begin{equation} \label{eq_v_all_round_sets}
	2^{\mu_\mathcal{A}} \vert \alpha_\mathcal{A} \vert < 2^{\mu'_B} \vert \alpha'{_B} \vert
\end{equation}

Therefore we have proven that $2^{\mu'_B} \vert \pi_B \upchain^{\mu'_B} \vert > 2^{\mu_\mathcal{A}} \vert \pi_\mathcal{A}^{\mu_\mathcal{A}} \vert$. From the definition of $\mu_B$, we know that $2^{\mu_B} \vert \pi_B \upchain^{\mu_B} \vert > 2^{\mu'_B} \vert \pi_B \upchain^{\mu'_B} \vert$ because it was chosen $\mu_B$ as level of comparison by the Verifier. So we conclude that $2^{\mu_B} \vert \pi_B \upchain^{\mu_B} \vert > 2^{\mu_\mathcal{A}} \vert \pi_\mathcal{A} \upchain^{\mu_\mathcal{A}} \vert$.
\end{proof}
