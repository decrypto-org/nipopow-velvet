\section{Analysis}
\begin{theorem}[Suffix Proofs Security under velvet fork]
	Assuming honest majority under velvet fork conditions (\ref{defn:velvet_honest_majority}) such that $t \leq (1 - \delta) \dfrac{n_h}{2}$ where $n_h$ the number of upgraded honest players, the non-interactive proofs-of-proof-of-work construction for computable k-stable monotonic suffix-sensitive predicates under velvet fork conditions in a typical execution is secure.
\textit{Proof.} By contradiction. Let $Q$ be a k-stable monotonic suffix-sensitive chain predicate. Assume for contradiction that NIPoPoWs under velvet fork on $Q$ is insecure. Then, during an execution at some round  $r_3$, $Q(\mathcal{C})$ is defined and the verifier $V$ disagrees with some honest participant. $V$ communicates with adversary $\mathcal{A}$ and honest prover $B$. The verifier receives proofs $\pi_\mathcal{A}, \pi_B$ which are of valid structure. Because $B$ is honest, $\pi_B$ is a proof constructed based on underlying blockchain $\mathcal{C}_B$ (with $\pi_B \subseteq \mathcal{C}_B$), which $B$ has adopted during round $r_3$ at which $\pi_B$ was generated. Consider $\widetilde{\mathcal{C}}_\mathcal{A}$ the set of blocks defined as $\widetilde{C}_\mathcal{A} = \pi_\mathcal{A} \cup \{ \bigcup \{\mathcal{C}_h^r\{:b_\mathcal{A}\}:  b_\mathcal{A} \in \pi_\mathcal{A}, \exists h,r : b_\mathcal{A} \in \mathcal{C}_{h}^{r}\}  \}$ where $\mathcal{C}_h^r$ the chain  that the honest player $h$ has at round $r$.
\end{theorem}

The verifier outputs $\neg Q(\mathcal{C}_B)$. Thus it is necessary that $\pi_\mathcal{A} {\geq}_m \pi_B$. We show that $\pi_\mathcal{A} {\geq}_m \pi_B$ is a negligible event.

Let the levels of comparison decided by the verifier be $\mu_\mathcal{A}$ and $\mu_B$ respectively. Let $b_0 = LCA(\pi_\mathcal{A}, \pi_B)$. Let $\mu'_B$ be the adequate level of proof $\pi_B$  with respect to block $b_0$. Call $\alpha_\mathcal{A} = \pi_\mathcal{A} \upchain^{\mu_\mathcal{A}}\{b_0:\}$, $\alpha'_B = \pi_B \upchain^{\mu'_B}\{b_0:\}$.

From Corollary \ref{cor:adversarial_proof_scheme} we have that the adversarial proof consists of a smooth interlink subchain followed by a thorny interlink subchain. We will refer to the smooth part of $\alpha_\mathcal{A}$ as $\alpha^{\mathcal{S}}_\mathcal{A}$ and to the thorny part as $\alpha^{\mathcal{T}}_\mathcal{A}$.

Our proof construction is based on the following intuition: we consider that $\alpha_\mathcal{A}$ consists of three distinct parts $\alpha_\mathcal{A}^1, \alpha_\mathcal{A}^2, \alpha_\mathcal{A}^3$ with the following properties.

Consider $b_0 = LCA(\pi_\mathcal{A}, \pi_B)$ the fork point between $\pi_\mathcal{A} \upchain^{\mu_\mathcal{A}}, \pi_B \upchain^{\mu_B}$ and $b_1 = LCA (\alpha^{\mathcal{S}}_\mathcal{A}, \mathcal{C}_B)$ the fork point between $\pi^{\mathcal{S}}_\mathcal{A} \upchain^{\mu_\mathcal{A}}, \mathcal{C}_B $ at the zero level as the honest prover could observe. Part
$\alpha_\mathcal{A}^1$ contains the blocks between $b_0$ exclusive and $b_1$ inclusive generated during the set of consecutive rounds $\mathcal{S}_1$ and $\vert  \alpha_\mathcal{A}^1 \vert = k_1$.
Consider $b_2$ the last block in $\alpha_\mathcal{A}$ generated by an honest player. Part $\alpha_{\mathcal{A}}^2$ contains the blocks between $b_1$ exclusive and $b_2$ inclusive generated during the set of consecutive rounds $\mathcal{S}_2$ and $\vert  \alpha_\mathcal{A}^2 \vert = k_2$. Consider $b_3$ the next block of $b_2$ in $\alpha_\mathcal{A}$. Then $\alpha_{\mathcal{A}}^3 = \alpha_\mathcal{A}[b_3:]$ and $\vert  \alpha_\mathcal{A}^3 \vert = k_3$ of all adversarially generated blocks generated during the set of rounds $\mathcal{S}_3$. So, $\vert \alpha_\mathcal{A} \vert = k_1 + k_2 + k_3$ and we will show that $\vert \alpha_\mathcal{A} \vert < \vert \alpha_B \vert$ .

The above are illustrated, among other, in Parts I, II of Figure \ref{fig:proof_velvet}.

\begin{figure}[h!]
	\begin{center}
    \includegraphics[scale=0.65]{figures/proof_velvet-crop.pdf}
	\end{center}
	\caption{\textit{ Wavy lines imply one or more blocks. Dashed lines and arrows imply interlink pointers to superblocks. \textbf{I}: the three round sets in two competing proofs at different levels, \textbf{II}: the corresponding 0-level blocks implied by the two proofs, \textbf{III}: blocks participating in chain $\mathcal{C}_B$ and block set $\widetilde{\mathcal{C}}_\mathcal{A}$ from the verifier's perspective.}}
    \label{fig:proof_velvet}
\end{figure}

We will now show three successive claims under velvet fork conditions: First that $\alpha_\mathcal{A}^1$ contains only a few blocks. Second,  $\alpha_\mathcal{A}^2$ contains only a few blocks. And third, the adversary is able to produce a winning $a_\mathcal{A}$ with negligible probability.\\

\textbf{Claim 1:} $\alpha_\mathcal{A}^1 = \alpha_\mathcal{A}(b_0 : b_1]$ contains only a few blocks. We have defined $b_0 = LCA(\pi_\mathcal{A}, \pi_B)$ and $b_1 = LCA(\alpha^{\mathcal{S}}_\mathcal{A}, \alpha'_B \downarrow)$. First observe that there are no thorny blocks in $\alpha_\mathcal{A}^1$ since $\alpha_\mathcal{A}^1[-1] = b_1$ is a smooth block. This means that if $b_1$ was generated at round $r_{b_1}$ and $\alpha^{\mathcal{S}}_\mathcal{A}[-1]$ in round $r$, then $r \geq r_{b_1}$. So, $\alpha_\mathcal{A}^1$ consists a valid chain. We show the statement considering the two possible cases for the relation of $\mu_\mathcal{A}, \mu'_B$.

\textit{\underline{Claim 1a}:} If $\mu'_B \leq \mu_A$ then they are completely disjoint. In such a case of inequality, every block in $\alpha_A$ would also be of lower level $\mu'_B$. Because of the adequate level $\mu'_B$ we know that $C\{b:\}\upchain^{\mu'_B} = \pi\{b:\}\upchain^{\mu'_B}$\cite{nipopows}. Subsequently, any block in $\pi_A\upchain^{\mu_A}\{b:\}[1:]$ would also be included in proof $\alpha'_B$, but $b=LCA(\pi_A, \pi_B)$ so there can be no succeeding block common in $\alpha_A, \alpha'_B$.

\textit{\underline{Claim 1b}:} If  $\mu'_B > \mu_A$ then $\vert \alpha_A[1:]  \cap \alpha'_B\downarrow[1:] \vert = k_1 \leq g(2^{\mu'_B - \mu_A})$.\\
Let's call $b$ the first block in $\alpha'_B$ after block $b_0$. Suppose for contradiction that $k_1 > g(2^{\mu'_B - \mu_A})$.  Since block $b$ of level $\mu'_B$ is also of level $\mu_A$, the adversary could include it in the proof but $b$ cannot exist in both $\alpha_A, \alpha'_B$ since $\alpha_A \cap \alpha'_B = \emptyset$ by definition. In case that the adversary chooses not to include $b$ in the proof then she can include no other blocks of $\mathcal{C}_B$ in her proof, since it would not consist a valid chain. Therefore, the adversary can include at most the $\mu_\mathcal{A}$ upgraded  blocks  between  $b_0$, $b$, which are expected to be equal to $g(2^{\mu'_B - \mu_A})$.

We conclude that $\vert \alpha^{\mathcal{S}}_\mathcal{A} \cap \alpha'_B\downarrow[1:] \vert = k_{1} \leq g(2^{\mu'_B - \mu_\mathcal{A}}) $, where $g$ the velvet parameter denoting the percentage of upgraded honest parties.

Consequently, there are at least $\vert\alpha_\mathcal{A}\vert - k_1$ blocks after block $b$ in $\alpha_\mathcal{A}$ which are not honestly generated blocks existing in $\mathcal{C}_B$. In other words, there are $\vert \alpha_\mathcal{A} \vert - k_1$ blocks after block $b$ in $\alpha_\mathcal{A}$, which are either thorny blocks existing in $\mathcal{C}_B$ either don't belong in $\mathcal{C}_B$.\\

\textbf{Claim 2.}
Part $\alpha_\mathcal{A}^2 = \alpha_\mathcal{A}(b_1:b_2]$ consists of only a few blocks. Let $ \vert \alpha_\mathcal{A}^2 \vert = k_2$. We have defined $b_2 = \alpha_\mathcal{A}^2[-1]$ to be the last block generated by an honest player in $\alpha_\mathcal{A}$. Consequently no thorny block exists in $\alpha_\mathcal{A}^2$, so all blocks in this part belong in a proper zero-level chain $\mathcal{C}_\mathcal{A}^2$.  Let $r_{b_1}$ be the round at which $b_1$ was generated. Since $b_1$ is the last block in $\alpha_\mathcal{A}$ which belongs in $\mathcal{C}_B$, then $\mathcal{C}_\mathcal{A}^2$ is a fork chain to $\mathcal{C}_B$ at some block $b'$ generated at round $r' \geq r_{b_1}$.

Let $r_2$ be the round when $b_2$ was generated by an honest party. Because an honest party has chain $\mathcal{C}_B$ at later round $r_3$ when the proof $\pi_B$ is constructed and because of the Common Prefix property on parameter $k_{2\downarrow} = g \cdot 2^{\mu_\mathcal{A}} \cdot k_2$, we conclude that $k_2 \leq k$, where $k$ is the Common Prefix parameter as defined in the Backbone series of papers\cite{backbone}.
% XXX: move definition of $k$ into preliminaries section

\textbf{Claim 3.} The adversary may submit a suffix proof such that $\alpha_\mathcal{A} \geq \alpha_B$ with negligible probability.
As argued earlier the last $k_3$ blocks included in $\alpha_\mathcal{A}$ are all thorny blocks. In the worst case  all $k_3 $ blocks are sewed from $\mathcal{C}_B$. This is the worst case scenario since each adversarially generated block in $\mathcal{C}_B$ may have dropped one smooth block out of the chain because of selfish mining. Considering this scenario, because of the strengthened Honest Majority Assumption for $(1/3)$-bounded adversary, Theorem 3 for Chain Quality guarantees that the majority of the blocks in $\mathcal{C}_B$ was computed by honest parties, thus the honestly generated blocks in $\mathcal{C}_B$ for the same round set sum to more amount of hashing power.\\
From all the above Claims we have that:\\
In the first round set, because of the common underlying chain:
\begin{equation} \label{eq_v_round_set_1}
2^{\mu_\mathcal{A}} \vert \alpha_\mathcal{A}^{k_1} \vert \leq 2^{\mu'_B} \vert \alpha'{_B^{k_1}} \vert
\end{equation}
Because of the adoption by an honest party of chain $\mathcal{C}_B$ at a later round $r_3$, we have for the second round set:
\begin{equation} \label{eq_v_round_set_2}
	2^{\mu_\mathcal{A}} \vert \alpha_\mathcal{A}^{k_2} \vert \leq 2^{\mu'_B} \vert \alpha'{_B^{k_2}} \vert
\end{equation}
In the third round set, because of good Chain Quality under the strengthened Honest Majority Assumption and Theorem 3 we have:
\begin{equation} \label{eq_v_round_set_3}
	2^{\mu_\mathcal{A}} \vert \alpha_\mathcal{A}^{k_3} \vert < 2^{\mu'_B} \vert \alpha'{_B^{k_3}} \vert
\end{equation}
Consequently we have:

\begin{equation*}
	2^{\mu_\mathcal{A}} ( \vert \alpha_\mathcal{A}^{k_1} \vert + \vert \alpha_\mathcal{A}^{k_2} \vert + \vert \alpha_\mathcal{A}^{k_3} \vert ) < 2^{\mu'_B} ( \vert \alpha'{_B^{k_1}} \vert + \vert \alpha'{_B^{k_2}} \vert + \vert \alpha'{_B^{k_3}} \vert) \Rightarrow
\end{equation*}

\begin{equation} \label{eq_v_all_round_sets}
	2^{\mu_\mathcal{A}} \vert \alpha_\mathcal{A} \vert < 2^{\mu'_B} \vert \alpha'{_B} \vert
\end{equation}

Therefore we have proven that $2^{\mu'_B} \vert \pi_B \upchain^{\mu'_B} \vert > 2^{\mu_\mathcal{A}} \vert \pi_\mathcal{A}^{\mu_\mathcal{A}} \vert$. From the definition of $\mu_B$, we know that $2^{\mu_B} \vert \pi_B \upchain^{\mu_B} \vert > 2^{\mu'_B} \vert \pi_B \upchain^{\mu'_B} \vert$ because it was chosen $\mu_B$ as level of comparison by the Verifier. So we conclude that $2^{\mu_B} \vert \pi_B \upchain^{\mu_B} \vert > 2^{\mu_\mathcal{A}} \vert \pi_\mathcal{A} \upchain^{\mu_\mathcal{A}} \vert$.

\begin{flushright}
	$\square$
\end{flushright}
