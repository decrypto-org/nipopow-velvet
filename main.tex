\documentclass[sigconf, anonymous]{acmart}

%\usepackage[colorlinks=true,citecolor=magenta,linkcolor=blue]{hyperref}
\usepackage{preamble}

\theoremstyle{plain}
\newtheorem{thm}{Theorem}[]
\theoremstyle{definition}
\newtheorem{defn}{Definition}
%\theoremstyle{lemma}
%\newtheorem{lemma}{Lemma}
%\theoremstyle{corollary}
%\newtheorem{corollary}{Corollary}

\fancyhf{} % Remove fancy page headers
\fancyhead[C]{Anonymous submission \#9999 to ACM CCS 2020} % TODO: replace 9999 with your paper number
\fancyfoot[C]{\thepage}

\setcopyright{none} % No copyright notice required for submissions
\acmConference[Anonymous Submission to ACM CCS 2020]{ACM Conference on Computer and Communications Security}{Due 15 May 2020}{London, TBD}
\acmYear{2020}

\settopmatter{printacmref=false, printccs=true, printfolios=true} % We want page numbers on submissions

%%\ccsPaper{9999} % TODO: replace with your paper number once obtained
% andri: at most 12 pages total

\begin{document}
\title{%Non-Interactive Proofs of Proof-of-Work under Velvet Fork
The Velvet Path to Superlight Blockchain Clients
} % TODO: replace with your title

\begin{abstract}
{\em Superlight} blockchain clients
learn facts about the blockchain state
while requiring merely polylogarithmic communication in the total 
number of blocks. 
For proof-of-work blockchains 
two known constructions exist: Superblock and FlyClient. 
%
Unfortunately, none of them can be deployed to existing blockchains as  they require consensus changes  and at least a soft fork to implement.

In this paper, we
investigate how a blockchain can be upgraded to support superblock clients without a soft fork. We show that it is possible to implement the needed changes without modifying the consensus protocol and by requiring only a minority of miners to upgrade, a process termed a ``velvet fork'' in the literature. While previous work conjectured that superblock clients can be safely deployed using velvet forks as-is, we show that previous constructions are insecure, and that using velvet techniques to interlink a blockchain can pose insidious security risks. We describe a novel class of attacks, called  ``chain-sewing'', which arise in the velvet fork setting: an adversary can cut-and-paste portions of various chains from independent forks, sewing them together to 
fool a superlight client into accepting a false claim.
We show how previous velvet fork constructions can be attacked via chain-sewing. 
Next we put forth the first provably secure velvet superblock client construction which we show  secure against adversaries that are bounded by 1/4 of the upgraded honest miner population. 
Like non-velvet superlight clients, our approach allows proving generic predicates about chains using infix proofs and as such can be adopted in practice for fast synchronization of transactions and accounts.
\end{abstract}

% TODO: replace this section with code generated by the tool at https://dl.acm.org/ccs.cfm
\begin{CCSXML}
    <ccs2012>
    <concept>
    <concept_id>10002978.10002979</concept_id>
    <concept_desc>Security and privacy~Cryptography</concept_desc>
    <concept_significance>300</concept_significance>
    </concept>
    </ccs2012>
\end{CCSXML}

\ccsdesc[300]{Security and privacy~Cryptography}

%\ccsdesc{Security and privacy~Use https://dl.acm.org/ccs.cfm to generate actual concepts section for your paper}
% -- end of section to replace with generated code

\keywords{blockchain, consensus, lightclients, NIPoPoW}

\maketitle

\section{Introduction}
Blockchains such as Bitcoin~\cite{nakamoto} and
Ethereum~\cite{buterin,wood} maintain chains of blocks that grow linearly
with time. A node synchronizing with the rest for the first time must download and validate the whole
chain if it does not rely on a trusted third party~\cite{taxonomy}. While a lightweight
node (SPV) can avoid downloading transactions beyond their
interest, it must still download block headers containing the
proof-of-work~\cite{pow} (PoW) of each block to determine which chain contains
the most work. Block headers, while smaller by a significant constant
factor, still grow linearly with time. An Ethereum node synchronizing for the
first time must download $4$ GB of block headers for
proof-of-work verification, even if it does not download
transactions. This has become a central problem to the usability of blockchain
systems for vendors who use mobile phones to accept payments
and sit behind limited internet bandwidth. They are forced to make a difficult
choice between decentralization and the ability to start accepting payments in a
timely manner.

Towards the goal of alleviating the burden of this download for SPV clients, a
number of \emph{superlight} clients has emerged.
%These clients are able to
%choose the best proof-of-work chain by only requesting a small number of
%\emph{sample} block headers instead of all the block headers.
These protocols give rise to  Non-Interactive Proofs of Proof-of-Work (NIPoPoW)
~\cite{nipopows}, short strings
 that ``compress'' the proof-of-work information of the chain by sending a selected sample of block headers.
%It has been shown
%that the block headers in these proofs are secure representatives of the
%proof-of-work of the underlying chain:
The necessary security property of such proofs is that a
minority adversary can only convince a
NIPoPoW client that a certain transaction is confirmed, only if she can
convince an SPV client, too.

There are two general directions for superlight clients: In the
\emph{superblock}~\cite{nipopows,compactsuperblocks} approach, the client
relies on \emph{superblocks}, blocks achieving much better
proof-of-work than required. In the
\emph{FlyClient}~\cite{flyclient} approach, blocks are randomly sampled and committed as in a $\Sigma$-protocol~\cite{schnorr} and a non-interactive
proof is calculated using Fiat--Shamir~\cite{fiatshamir}. The number of block headers
that must be sent then grows only logarithmically with time. The NIPoPoW
client, which is the proof \emph{verifier} here, relies on a connection to full nodes,
who, acting as \emph{provers},  perform block sampling from the full
chain.
No trust assumptions are made for these provers, as the
verifier can check the veracity of their claims. As long as the verifier is
connected to at least one honest prover (an assumption also made in the SPV
protocol~\cite{eclipse,eclipse-ethereum}), they arrive at the correct chain.

In both approaches, the verifier must check that blocks
sampled one way or another were generated in the same order as presented by the prover. As such, each block in the proof contains a
pointer to the previous block in the proof. As blocks in these proofs are far
apart in the underlying blockchain, the legacy \emph{previous block pointer}
of block headers does not suffice.
Both approaches require modifications to the consensus layer.
For superblock NIPoPoWs, the block header is
modified to include, in addition to a pointer to the previous block, pointers to
a small amount of recent high-proof-of-work blocks. For FlyClient,
each block additionally contains pointers to all previous blocks in the
chain. Both these modifications can be made efficiently by organizing these
pointers into Merkle Trees~\cite{merkle} or Mountain Ranges~\cite{ct,mmr}
whose root is stored in the header. Including such extra pointers in
blocks is termed \emph{interlinking the chain}~\cite{popow}.

The modified block format must be respected
and validated by full nodes and thus requires a hard or soft fork. However, even soft forks require a supermajority approval, and features considered non-essential by the community have
taken years to receive it~\cite{segwit}. Towards the goal of implementing
superlight clients sooner, we study the question of whether
superlight clients can be deployed without soft forks. We propose some
\emph{helpful but untrusted} modifications to blocks. These
mandate that some extra data is included in each block. The
data is placed inside the block by upgraded miners only, while the rest of the
miners do not include this data into their blocks and do not verify
its inclusion, treating them merely as comments. To maintain backwards
compatibility, contrary to a soft fork, upgraded miners accept blocks
produced by unupgraded miners that
do not contain the extra data, or
even blocks containing invalid or malicious such data produced by a
mining adversary. This acceptance is necessary to avoid causing a chain
split with the unupgraded nodes. Such a modification to
consensus is termed a \emph{velvet fork}~\cite{velvet}.
A summary of our contributions is as follows:
\begin{enumerate}
  \item We illustrate that, contrary to previous claims, superlight
        clients designed to work in a soft fork cannot be readily deployed in a velvet fork. We present the novel and insidious
        \emph{chain-sewing} attack which thwarts defenses of previous proposals and allows a minority adversary to cause
        catastrophic failures.
  \item We propose the first \emph{backwards-compatible superlight client}. We put forth an interlinking mechanism implementable through a velvet fork. We construct a superblock NIPoPoW protocol on top of it and build clients for statements about the blockchain state via ``suffix'' and ``infix'' proofs.
  \item We prove our construction secure in the synchronous static difficulty model against adversaries bounded to $1/3$ of the mining power of the honest upgraded nodes. Our protocol works even if a minority adopts it.
\end{enumerate}

\noindent
\textbf{Previous work.} Proofs of Proof-of-Work were proposed in the
context of superlight clients~\cite{nipopows,flyclient},
cross-chain communication~\cite{pow-sidechains,burn,crosschain-sok}, and
local data consumption by smart contracts~\cite{derivatives}. Interlinking
has been deployed in production both since genesis~\cite{ergo,nimiq} and using
hard forks~\cite{heartwood-flyclient}, and relevant verifiers have been
implemented~\cite{gglou,nipopow-gas}. They have
been conjectured to work in velvet fork conditions~\cite{nipopows} (we show
here that these conjectures are ill-informed in light of our attack). Velvet forks have seen many other
applications~\cite{velvet} and have been deployed in practice~\cite{gtklocker}. In this work,
we focus on consensus state compression. Alternative constructions in the hard-fork setting use zk-SNARKS~\cite{coda} or concern
Proof-of-Stake~\cite{pos-sidechains}. Complementary to consensus state
compression (i.e., the compression of block headers) is
the compression of application state, namely the State Trie, UTXO, or
transaction history. A series of works complementary and composable with ours
discusses this~\cite{edrax,ethanos}.

\noindent
\textbf{Organization.}
In Section~\ref{sec:preliminaries}, we describe the known superblock NIPoPoW protocol designed for a soft fork (for a detailed description refer to Appendix~\ref{sec:nipopows_protocol} and \cite{nipopows}).
%give a brief overview of the \emph{backbone model} and its notation; as we
%heavily leverage the machinery of the model, the section is a necessary
%prerequisite to follow the analysis. The rest of the paper is structured in the
%form of \emph{a proof and refutation}~\cite{lakatos}. We believe this form is
%more digestible.
In Section~\ref{sec:velvet}, we discuss the velvet model and
some initial definitions, and present a first attempt towards a velvet
NIPoPoW scheme which appeared in previous work.
%We discuss the informal argument of why it seems to be secure, which we \emph{refute} with
A concrete attack is explored
in Section~\ref{sec:attack} (we give simulation results and concrete parameters
for our attack in Appendix~\ref{sec:simulation}). In Section~\ref{sec:construction}, we patch the
scheme and put forth our more elaborate and novel Velvet NIPoPoW construction.
We analyze it and formally prove it secure in Appendix~\ref{sec:analysis}. Our scheme at this
point allows verifiers to decide which blocks form a \emph{suffix} of the
longest blockchain and thus the protocol supports \emph{suffix proofs}. We
extend our scheme to allow any block of interest within the chain to be
demonstrated to a prover in a straightforward manner in
Section~\ref{sec:infix}, giving a full \emph{infix proof} protocol. The latter
protocol can be deployed today in real blockchains,
including Bitcoin, to confirm payments achieving both decentralization and
timeliness, solving a major outstanding dilemma in contemporary blockchain
systems.

\subsection{Velvet fork parameter}
A velvet fork suggest that only a minority of upgraded parties needs to support the protocol changes. Let $g$ express the percentage of honest upgraded parties to the total number of miners. We will refer to $g$ as the ``velvet parameter".

\begin{definition}[Velvet Parameter]
	\textit{Let $g$ be the velvet parameter for NIPoPoW protocols. Then if $n_h$ the upgraded honest miners and $n$ the total number of miners $t$ out of which are corrupted, it holds that $n_h = g (n - t)$.}
	\label{defn:velvet_honest_majority}
\end{definition}

\subsection{Smooth and Thorny blocks}
In order to be applied under velvet fork, a protocol has to change in a backwards-compatible manner. In essence, any additional information coming with the protocol upgrade is transparent to the non-upgraded players. This transparency towards the non-upgraded parties requires any block that conforms only to the old protocol rules to be considered a valid one. Considering superblock NIPoPoWs under a velvet fork, any block is to be checked for its validity regardless the validity of the NIPoPoWs protocol's additional information, which is the interlink structure.

A block generated by the adversary could thus contain arbitrary data in the interlink and yet be appended in the chain adopted by an honest party. In case that trash data are stored in the structure this could be of no harm for the protocol routines, since such blocks will be treated as non-upgraded. In the context of the attack that will be presented in the following section, we examine the case where the adversary includes false interlink pointers. 

An interlink pointer is the hash of a block. A correct interlink pointer of a block $b$ for a specific level $\mu$ is a pointer to the most recent $b$'s ancestor of level $\mu$. From now on we will refer to correct interlink pointers as \emph{smooth pointers}. Pointers of the 0-level \textit{(prevIds)} are always smooth because of the performed proof-of-work.

\begin{definition}[Smooth Pointer]
    \textit{Smooth pointer of a block $b$ for a specific level $\mu$ is the     interlink pointer to the most recent $\mu$-level ancestor of $b$.}
	\label{defn:smooth_pointer}
\end{definition}


A non-smooth pointer may not point to the most recent ancestor of level $\mu$ or even point to a superblock of a fork chain, as shown in Figure \ref{fig:false_interlink}.

\begin{figure}[h]
	\begin{center}
		\includegraphics[scale=0.7]{figures/false_interlink.pdf}
	\end{center}
    \caption{\textit{A non-smooth pointer of an adversarial block, colored  black, in an honest player's chain.}}
	\label{fig:false_interlink}
\end{figure}

In the same manner it is possible that a false interlink contain arbitrary pointers to blocks of any chain as illustrated in Figure \ref{fig:thorny_block}. The interlink pointing to arbitrary directions resembles a thorny bush, so we will refer to blocks containing false interlink information as \emph{thorny}.

\begin{definition}[Thorny Block]
	\textit{Thorny block is a block which contains at least one non-smooth interlink pointer.}
	\label{defn:thorny_block}
\end{definition}

\begin{figure}[h]
	\begin{center}
		\includegraphics[scale=0.75]{figures/thorny_block.pdf}
	\end{center}
	\caption{\textit{A thorny block appended in an honest player's chain. The dashed arrows 	are interlink pointers.}}
	\label{fig:thorny_block}
\end{figure}

Opposite of the thorny are the \emph{smooth} blocks, which may be blocks generated by non-upgraded players or blocks generated by upgraded players and contain only smooth pointers in their interlink.

\begin{definition}[Smooth Block]
	\textit{Smooth block is any block which is not thorny.}
	\label{defn:smooth_block}
\end{definition}
\section{Velvet Interlinks}\label{sec:velvet}
Velvet forks were recently introduced~\cite{velvet}. In a
velvet fork, blocks created by upgraded miners (called \emph{velvet blocks}) are
accepted by unupgraded miners as in a soft fork. Additionally, blocks created by
unupgraded miners are also accepted by upgraded miners. This allows the protocol
to upgrade even if only a minority of miners upgrade. To maintain
backwards compatibility and to avoid causing a permanent fork, the additional data included
in a block is \emph{advisory} and must be accepted whether it exists or not.
Even if the additional data is invalid or malicious, upgraded nodes (in this
context also called \emph{velvet nodes}) are forced to accept the blocks. The
simplest approach to interlink the chain with a velvet fork is to have
upgraded miners include the interlink pointer in the coinbase of the blocks they produce, but
accept blocks with missing or incorrect interlinks. As we show in the next
section, this approach is flawed and susceptible to unexpected attacks. A
surgical change in the way velvet blocks are produced is necessary to achieve
security.

In a velvet fork, only a minority of honest parties needs to support the protocol
changes. We refer to this percentage as the ``velvet parameter''.

\begin{definition}[Velvet Parameter]
	The \emph{velvet parameter} $g$ is defined as the percentage of honest parties
	that have upgraded to the new protocol. The absolute number of honest upgraded
	parties is denoted $n_h$ and it holds that
	$n_h = g (n - t)$.
	\label{defn:velvet_honest_majority}
\end{definition}

Unupgraded honest nodes will produce blocks
containing no interlink, while upgraded honest nodes will produce blocks
containing truthful interlinks. Therefore, any block with invalid interlinks is
adversarial. However, such blocks cannot be rejected by the
upgraded nodes, as this gives the adversary an opportunity to cause a permanent
fork. A block generated by the adversary can thus contain arbitrary interlinks
and yet become honestly adopted. Because the honest prover is an
upgraded full node, it determines what the correct interlink pointers are by
examining the whole previous chain, and can deduce whether a block contains
invalid interlinks. In that case, the prover can simply treat such blocks as
unupgraded. In the context of the attack presented in the following
section, we examine the case where the adversary includes false interlink
pointers. We distinguish blocks based on whether they follow the velvet protocol
rules or they deviate from them.

\begin{definition}[Smooth and Thorny blocks]
A block in a velvet upgrade is called \emph{smooth} if it contains
auxiliary data corresponding to the honest upgraded protocol. A block
is called \emph{thorny} if it contains auxiliary data, but the data differs from
the honest upgraded protocol. A block is neither smooth nor thorny if it
contains no auxiliary data.
\end{definition}

In the case of velvet forks for interlinking, the auxiliary data consists
of the interlink Merkle Tree root.\\
\noindent
\textbf{A na\"ive velvet scheme.}
In previous work~\cite{nipopows}, it was conjectured that superblock NIPoPoWs
remain secure under a velvet fork. We call this scheme the \emph{Na\"ive Velvet
NIPoPoW} protocol. It is not dissimilar from the NIPoPoW protocol in the
soft fork case. The na\"ive velvet NIPoPoW protocol works as follows.
Each upgraded honest miner attempts to mine a block $b$
that includes interlink pointers in the form of a Merkle Tree included in its
coinbase transaction. For each level $\mu$, the interlink contains a pointer to
the most recent among all the ancestors of $b$ that have achieved at least
level $\mu$, regardless of whether the referenced block is upgraded or not and
regardless of whether its interlinks are valid. Unupgraded honest nodes will
keep mining blocks on the chain as usual; because the status of a block as
superblock does not require it to be mined by an upgraded miner, the unupgraded
miners contribute mining power to the creation of superblocks.

The prover in the na\"ive velvet NIPoPoWs works as follows. The honest
prover constructs the proof $\pi \chi$ as in Algorithm~\ref{alg.nipopow-maxchain}.
The outstanding issue is that
$\pi$ does not form a chain because some of
its blocks may not be upgraded and they may not contain any pointers (or may
contain invalid pointers).
Suppose $\pi[i]$ is the most recent $\mu$-superblock preceding $\pi[i + 1]$.
The prover must provide a connection between $\pi[i + 1]$ and $\pi[i]$.
The block $\pi[i + 1]$ is a superblock and exists at some position $j$ in the
underlying chain $\chain$ of the prover, i.e., $\pi[i + 1] = \chain[j]$. If
$\chain[j]$ is smooth, then the interlink pointer at level $\mu$ within
it can be used. Otherwise, the prover uses the \emph{previd} pointer of
$\pi[i + 1] = \chain[j]$ to repeatedly reach the parents of $\chain[j]$, namely
$\chain[j - 1], \chain[j - 2], \cdots$ until a smooth block $b$ between $\pi[i]$
and $\pi[i + 1]$ is found in $\chain$, or until $\pi[i]$ is reached.
The block $b$ contains a pointer to
$\pi[i]$, as $\pi[i]$ is also the most recent $\mu$-superblock ancestor of $b$.
The blocks $\chain[j - 1], \chain[j - 2], \cdots, b$ are then included in the
proof to illustrate that $\pi[i]$ is an ancestor of $\pi[i + 1]$.
The argument for why this scheme works can be found in Appendix~\ref{naive_velvet_scheme}.

\begin{figure}[H]
	\begin{center}
		\iftwocolumn
			\includegraphics[width=0.6\columnwidth]{figures/false_interlink.pdf}
		\else
			\includegraphics[width=0.35\columnwidth]{figures/false_interlink.pdf}
		\fi
	\end{center}
    \caption{A thorny block, colored black, in an honest party's chain, uses its interlink to point to a fork chain.}
	\label{fig:false_interlink}
\end{figure}

\section{The Chainsewing Attack}\label{sec:attack}
We now make the critical observation that a thorny block can include interlink
pointers to blocks that are not its own ancestors in the $0$-level chain.
Because it must contain a pointer to the hash of the block it points to, they
must be older blocks, but they may belong to a
different $0$-level chain. This is shown in Figure~\ref{fig:false_interlink}.
In fact, as the interlink vector contains multiple pointers, each pointer may
belong to a different fork. This is illustrated in
Figure~\ref{fig:thorny_block}. The interlink pointing to arbitrary directions
resembles a thorny bush.

\begin{figure}
	\begin{center}
		\iftwocolumn
			\includegraphics[width=0.65\columnwidth]{figures/thorny_block.pdf}
		\else
			\includegraphics[width=0.4\columnwidth]{figures/thorny_block.pdf}
		\fi
	\end{center}
	\caption{A thorny block appended to an honest party's chain.
	The dashed arrows are interlink pointers.}
	\label{fig:thorny_block}
\end{figure}

We now present the \emph{chainsewing attack} against the na\"ive velvet NIPoPoW
protocol. The attack leverages thorny blocks in order to enable the adversary to
\emph{usurp} blocks belonging to a different chain and claim them as her own.
Taking advantage of thorny blocks, the adversary produces suffix proofs
containing an arbitrary number of blocks belonging to several fork chains. The
attack works as follows.

Let $\chain_B$ be a chain adopted by an honest party $B$ and $\chain_\mathcal{A}$, a fork of $\chain_B$ at some point, maintained by the adversary. After the fork point $b = (\chain_B \cap \chain_\mathcal{A})[-1]$, the honest party produces a block extending $b$ in $\chain_B$ containing a transaction $\tx$. The adversary includes a conflicting (double spending) transaction $\tx'$ in a block extending $b$ in $\chain_\mathcal{A}$.
The adversary produces a suffix proof to convince the verifier that $\chain_\mathcal{A}$ is longer (even in the case $\chain_\mathcal{A}$ is not). In order to achieve this, the adversary needs to include a greater amount of total proof-of-work in her suffix proof, $\pi_\mathcal{A}$, in comparison to that included in the honest party's proof, $\pi_B$, so as to achieve $\pi_\mathcal{A} \geq_m \pi_B$. Towards this purpose, she mines intermittently on both $\chain_B$ and $\chain_\mathcal{A}$. She produces some thorny blocks in both chains $\chain_\mathcal{A}$ and $\chain_B$ which will allow her to usurp selected blocks of $\chain_B$ and present them to the light client as if they belonged to $\chain_\mathcal{A}$ in her suffix proof.

The general form of this attack for an adversary sewing blocks to one forked chain is illustrated in Figure~\ref{fig:generic_attack}. Dashed arrows represent interlink pointers of some level $\mu_\mathcal{A}$. Starting from a thorny block in the adversary's forked chain and following the interlink pointers, jumping between $\chain_\mathcal{A}$ and $\chain_B$, a chain of blocks crossing forks is formed, which the adversary claims as part of her suffix proof. Blocks of both chains are included in this proof and a verifier cannot distinguish the adversarial pointers participating in this proof chain and, as a result, considers it a valid proof. Importantly, the adversary must ensure that any blocks usurped from the honest chain are not included in the honest NIPoPoW to force the NIPoPoW verifier to consider an earlier LCA block $b$; otherwise, the adversary will compete after a later fork point, negating any sewing benefits.

\begin{figure}
	\begin{center}
		\includegraphics[width=0.95\columnwidth
		]{figures/generic_chainsewing_attack.pdf}
	\end{center}
	\caption{Generic Chainsewing Attack. $\chain_B$ is the chain of an honest party and $\chain_\mathcal{A}$ is the adversary's chain. Thorny blocks are colored black. Dashed arrows represent interlink pointers included in the	adversary's suffix proof. Wavy lines imply one or more blocks.}
	\label{fig:generic_attack}
\end{figure}

This generic attack is made concrete as follows.
The adversary chooses to attack at some level $\mu_\mathcal{A} \in \mathbb{N}$. Ideally, the adversary chooses $\mu_A$ to be as low as possible. As shown in Figure~\ref{fig:attack}, she first generates a block $b'$ in her forked chain $\chain_\mathcal{A}$ containing the double spend, and a block $a'$ in the honest chain $\chain_B$ which thorny-points to $b'$. Block $a'$ will be accepted as valid in the honest chain $\chain_B$ despite the invalid interlink pointers. The adversary also chooses a desired superblock level $\mu_B \in \mathbb{N}$ that she wishes the honest party to attain. Subsequently, the adversary waits for the honest party to mine and sews any blocks mined on the honest chain that are of level below $\mu_B$. However, she must bypass blocks that she thinks the honest party will include in their final NIPoPoW, which are of level $\mu_B$ (the blue block designated $c$ in Figure~\ref{fig:attack}). To bypass a block, the adversary mines her own thorny block $d$ on top of the current honest tip (which could be equal to the block to be bypassed, or have progressed further), containing a thorny pointer to the block preceding the block to be bypassed and hoping $d$ will not exceed level $\mu_B$ (if it exceeds that level, she discards her $d$ block). Once $m$ blocks of level $\mu_B$ have been bypassed in this manner, the adversary starts bypassing blocks of level $\mu_B - 1$, because the honest NIPoPoW will start including lower-level blocks. The adversary continues descending in levels until a sufficiently low level $\min\mu_B$ has been reached at which point it becomes uneconomical for the adversary to continue bypassing blocks (typically for a $1/4$ adversary, $\min\mu_B = 2$). At this point, the adversary forks off of the last sewed honest block. This last honest block will be used as the last block of the adversarial $\pi$ part of the NIPoPoW proof. She then independently mines a $k$-long suffix for the $\chi$ portion and creates her NIPoPoW $\pi \chi$. Lastly, she waits for enough time to pass so that the honest party's chain progresses sufficiently to make the previous bypassing guesses correct and so that no blocks in the honest NIPoPoWs coincide with blocks that have not been bypassed. This requires to wait for the following blocks to appear in the honest chain: $2m$ blocks of level $\mu_B$; after the $m^\text{th}$ $\mu_B$-level block, a further $2m$ blocks of level $\mu_B - 1$; after the $m^\text{th}$ such block, a further $2m$ blocks of the level $\mu_B - 2$, and so on until level $0$ is reached.

\begin{figure}
	\begin{center}
		\includegraphics[width=0.99\columnwidth]{figures/chainsew-concrete.pdf}
	\end{center}
	\caption{A portion of the concrete Chainsewing Attack. The adversary's blocks are shown in black, while the honestly generated blocks are shown in white. Block $b'$ contains a double spend, while block $a'$ sews it in place. The blue block $c$ is a block included in the honest NIPoPoW, but it is bypassed by the adversary by introducing block $d$ which, while part of the honest chain, points to $c$'s parent. After a point, the adversary forks off and creates $k = 3$ of their own blocks.}
	\label{fig:attack}
\end{figure}

In this attack the adversary uses thorny blocks to ``sew'' portions of the
honestly adopted chain to her own forked chain. This justifies the name given to
the attack.
In order to make this attack successful, the adversary needs only to
produce few superblocks, while she can arrogate a large number of
honestly produced blocks.
%Thus the attack succeeds with non-negligible probability.

The attack described against superblock NIPoPoWs generalizes to other superlight protocols.
We give an overview of the chainsewing attack against velvet FlyClient~\cite{flyclient} in
Appendix~\ref{sec:flyclient}.

%We illustrate simulation results for the success rate of our attack in Appendix~\ref{sec:simulation}. Our experiments find the attack with parameters $\mu_B = 10, \mu_A = 0, t = 1, n = 5, k = 15$ succeeds with a constant rate of success of approximately $0.26$, when the security parameter $m$ ranges from $3$ to $15$. This is in contrast to the best previously known attack (which does not make use of thorny blocks), which succeeds with probability less than $0.01$. Previous work recommends $m = 15$ for a $1/3$ adversary for a probability of failure bounded by $0.001$.

\section{Velvet NIPoPoWs}
In order to eliminate the Chainsewing Attack we propose an update to the velvet NIPoPoW protocol. The core problem is that in her suffix proof the adversary is able to claim not only blocks of forked chains,  which are in majority adversarially generated due to the Common Prefix property, but also arbitrarily long parts of the chain adopted by an honest player. Since thorny blocks are accepted as valid, the verifier cannot distinguish blocks that actually belong in a chain from blocks that only seem to belong in the same chain because they are pointed to via a non-smooth pointer.

We describe a protocol patch that operates as follows. The NIPoPoW protocol under velvet fork works as usual but each miner constructs smooth blocks. This means that  a block's interlink is constructed excluding thorny blocks. In this way, although thorny blocks are accepted in the chain, they are not taken into consideration when updating the interlink structure for the next block to be mined. No honest block could now point to a thorny superblock that may act as the passing point to the fork chain in an adversarial suffix proof. Thus, after this protocol update the adversary is only able to inject adversarially generated blocks from an honestly adopted chain to her own fork.
At the same time, thorny blocks cannot participate in an honestly
generated suffix proof except for some blocks in the proof's suffix $(\chi)$. This argument holds because thorny blocks do not form a valid chain along with honestly mined blocks anymore. Consequently, as far as the blocks included in a suffix proof is concerned, we can think of thorny blocks as belonging in the adversary's fork chain for the $\pi$ part of the proof,  which is the comparing part between proofs. Figure~
\ref{fig:injection} illustrates this remark.

\begin{figure}[h!]
	\begin{center}
		\includegraphics[scale=0.5]{figures/injection.png}
	\end{center}
	\caption{The adversarial fork chain $\chain_A$ and chain $\chain_B$ of an honest player. Thorny blocks are colored black. Dashed arrows represent interlink pointers. Wavy lines imply one or more blocks. After the protocol update, when an adversarially generated block is sewed from $\chain_B$ into the adversary's suffix proof the verifier conceives $\chain_A$ as longer and $\chain_B$ as shorter. \textbf{I:} The real picture of the chains. \textbf{II:} Equivalent picture from the verifier's perspective considering the blocks included in the corresponding suffix proof for each chain.}
	\label{fig:injection}
\end{figure}

In order to make NIPoPoWs a secure protocol under velvet fork conditions we suggest that the NIPoPoW protocol under velvet fork works as usual but a miner constructs a block's interlink without pointers to thorny blocks.

In order to prove the security of the updated protocol we need to strengthen the Honest Majority Assumption to 1/4 adversary with respect to upgraded honest miners.

\begin{definition}[Velvet Honest Majority]
	\textit{Let $n_h$ be the number of upgraded honest miners. Then $t$ out of total $n$ parties are corrupted such that $\dfrac{t}{n_h} < \dfrac{1 - \delta_v}{3} $.}
	\label{defn:velvet_honest_majority}
\end{definition}

The following Lemmas come as immediate results from the suggested protocol
update.

\begin{lemma}
	A velvet suffix proof constructed by an honest player cannot contain any thorny block.
	\label{lemm:smooth_honest_suffix}
\end{lemma}

\begin{lemma}
	Let $\mathcal{P_A} = (\pi_\mathcal{A}, \chi_\mathcal{A})$ be a velvet suffix proof constructed by the adversary and block $b_s$, generated at round $r_s$, be the most recent smooth block in the proof. Then $\forall r:r < r_s$ no thorny blocks generated at round $r$ can be included in $\mathcal{P_A}$.
	\label{lemm:smooths_before_smooth}
\end{lemma}
\begin{proof}
By contradiction. Let $b_t$ be a thorny block generated at some round $r_t < r_s$. Suppose for contradiction that $b_t$ is included in the proof. Then, because $\mathcal{P_A}$ is a valid chain as for interlink pointers, there exist a block path made by interlink pointers starting from $b_s$ and resulting to $b_t$. Let $b'$ be the most recently generated thorny block after $b_t$ and before $b_s$ included in $\mathcal{P_A}$. Then $b'$ has been generated at a round $r'$ such that $r_t \leq r' < r_s$. Then the block right after block $b'$ in $\mathcal{P_A}$ must be a thorny block since it points to $b'$ which is thorny. But $b'$ is the most recent thorny block after $b_t$, thus we have reached a contradiction.
\end{proof}

\begin{lemma}
	Let $\mathcal{P_A} = (\pi_\mathcal{A}, \chi_\mathcal{A})$ be a velvet suffix proof constructed by the adversary. Let $b_t$ be the oldest thorny bock included in $\mathcal{P_A}$ which is generated at round $r_t$. Then any block $b = \{b: b \in \mathcal{P_A} \wedge \text{b generated at }r \geq r_t \}$ is thorny.
	\label{lemm:thorny_after_thorny}
\end{lemma}
\begin{proof}
By contradiction. Suppose for contradiciton that $b_s$ is a smooth block generated at round $r_s > r_t$. Then from Lemma \ref{lemm:smooths_before_smooth} any block generated at round $r < r_s$ is smooth. But $b_t$ is generated at round $r_t < r_s$ and is thorny, thus we have reached a contradiction.
\end{proof}

The following corollary emerges immediately from Lemmas \ref{lemm:smooths_before_smooth}, \ref{lemm:thorny_after_thorny}. This result is illustrated in Figure \ref{fig:adversarial_velvet_proof}.

\begin{corollary}
	Any adversarial proof $\mathcal{P_A} = (\pi_\mathcal{A}, \chi_\mathcal{A})$ containing both smooth and thorny blocks consists of a prefix smooth subchain followed by a suffix thorny subchain.
	\label{cor:adversarial_proof_scheme}
\end{corollary}

\begin{figure}[h!]
	\begin{center}
		\includegraphics[scale=0.75]{figures/adversarial_velvet_proof.pdf}
	\end{center}
	\caption{In the general case the adversarial velvet suffix proof $\mathcal{P_A} = (\pi_\mathcal{A}, \chi_\mathcal{A})$ consists of an initial part of smooth blocks followed by thorny blocks.}
	\label{fig:adversarial_velvet_proof}
\end{figure}
We now describe the algorithms needed by the upgraded miner, prover and verifier. The upgraded miner acts as usual except for including the interlink of the newborn block in the coinbase transaction. In order to construct an interlink containing only the smooth blocks, the miner keeps a copy of the ``smooth chain'' ($\chain_S$) which consists of the smooth blocks existing in the original chain $\chain$. The algorithm for extracting the smooth chain out of $\chain$ is given in Algorithm \ref{alg:smooth_chain}. Function \textit{isSmoothBlock(B)} checks whether a block \textit{B} is a smooth velvet by calling \textit{isSmoothPointer(B,p)} for every pointer \textit{p} in \textit{B}'s interlink. Function \textit{isSmoothPointer(B,p)} returns \emph{true} if $p$ is a valid pointer, in essence a pointer to the most recent \emph{smooth velvet} for the level denoted by the pointer itself. The \textit{updateInterlink} algorithm is given in Algorithm \ref{alg:updateInterlink}, which is essentially the same as in the case of a hard/soft fork, except for working on the smooth chain $\chain_S$ instead of $\chain$.

The construction of the velvet suffix prover is given in Algorithm \ref{alg:velvet_suffix_prover}, which is essentially the same to that of a hard/soft fork except for working on smooth chain $\chain_S$ instead of $\chain$.

In conclusion the Verify algorithm for the NIPoPoW suffix protocol remains the same as in the case of hard or soft fork.

\begin{algorithm}[H]
	\caption{\label{alg:smooth_chain}The computation of a smooth chain}
	\begin{algorithmic}[1]
			\Function{\sf smoothChain}{$\chain$}
				\Let{\chain_S}{\{\mathcal{G}\}}
				\Let{k}{1}
				\While{$\chain[-k] \neq \mathcal{G}$}
						\If{$\textsf{isSmoothBlock}(\chain[-k])$}
							\Let{\chain_S}{\chain_S \cup \chain[-k]}
						\EndIf
						\Let{k}{k + 1}
				\EndWhile
				\State\Return{$\chain_S$}
			\EndFunction
	\end{algorithmic}
	\vspace{4mm}
	\begin{algorithmic}[1]
			\Function{\sf isSmoothBlock}{$B$}
				\If{$B = \mathcal{G}$}
					\State\Return{$\true$}
				\EndIf
				\For{$p \in B.\textsf{interlink}$}
						\If{$\lnot \textsf{isSmoothPointer}(B, p)$}
							\State\Return{$\false$}
						\EndIf
				\EndFor
				\State\Return{$\true$}
			\EndFunction
	\end{algorithmic}
	\vspace{4mm}
	\begin{algorithmic}[1]
			\Function{\sf isSmoothPointer}{$B, p$}
				\Let{b}{\textsf{Block}(B.\textsf{prevId})}
				\While{$b \neq p$}
						\If{$\textsf{level}(b) \geq \textsf{level}(p) \land \textsf{isSmoothBlock}(b)$}
							\State\Return{$\false$}
						\EndIf
						\If{$b = \mathcal{G}$}
							\State\Return{$\false$}
						\EndIf
						\Let{b}{\textsf{Block}(b.\textsf{prevId})}
				\EndWhile
				\State\Return{$\textsf{isSmoothBlock}(b)$}
			\EndFunction
	\end{algorithmic}
\end{algorithm}

\begin{algorithm}[h!]
		\caption{\label{alg:updateInterlink}Velvet updateInterlink}
		\begin{algorithmic}[1]
				\Function{\sf updateInterlinkVelvet}{$\chain_S$}
						\Let{B'}{\chain_S[-1]}
						\Let{\textsf{interlink}}{B'.\textsf{interlink}}
						\For{$\mu = 0$ to $\textsf{level}(B')$}
								\Let{\textsf{interlink}[\mu]}{\textsf{id}(B')}
						\EndFor
						\State\Return$\textsf{interlink}$
				\EndFunction
		\end{algorithmic}
\end{algorithm}

\begin{algorithm}[h!]
		\caption{\label{alg:velvet_suffix_prover}Velvet Suffix Prover}
		\begin{algorithmic}
				\Function{\sf ProveVelvet$_{m,k}$}{$\chain_S$}
					\Let{B}{\chain_S[0]}
					\For{$\mu = \lvert \chain_S[-k].\textsf{interlink} \rvert$ down to $0$}
						\Let{\alpha}{\chain_S[{:}-k]\{B{:}\}\upchain^\mu}
						\Let{\pi}{\pi \cup \alpha}
						\Let{B}{\alpha[-m]}
					\EndFor
					\Let{\chi}{\chain_S[-k{:}]}
					\State\Return$\pi\chi$
				\EndFunction
		\end{algorithmic}
\end{algorithm}

\section{Analysis}
\begin{theorem}[Suffix Proofs Security under velvet fork]
	Assuming honest majority under velvet fork conditions (\ref{defn:velvet_honest_majority}) such that $t \leq (1 - \delta) \dfrac{n_h}{2}$ where $n_h$ the number of upgraded honest players, the non-interactive proofs-of-proof-of-work construction for computable k-stable monotonic suffix-sensitive predicates under velvet fork conditions in a typical execution is secure.
\textit{Proof.} By contradiction. Let $Q$ be a k-stable monotonic suffix-sensitive chain predicate. Assume for contradiction that NIPoPoWs under velvet fork on $Q$ is insecure. Then, during an execution at some round  $r_3$, $Q(\mathcal{C})$ is defined and the verifier $V$ disagrees with some honest participant. $V$ communicates with adversary $\mathcal{A}$ and honest prover $B$. The verifier receives proofs $\pi_\mathcal{A}, \pi_B$ which are of valid structure. Because $B$ is honest, $\pi_B$ is a proof constructed based on underlying blockchain $\mathcal{C}_B$ (with $\pi_B \subseteq \mathcal{C}_B$), which $B$ has adopted during round $r_3$ at which $\pi_B$ was generated. Consider $\widetilde{\mathcal{C}}_\mathcal{A}$ the set of blocks defined as $\widetilde{C}_\mathcal{A} = \pi_\mathcal{A} \cup \{ \bigcup \{\mathcal{C}_h^r\{:b_\mathcal{A}\}:  b_\mathcal{A} \in \pi_\mathcal{A}, \exists h,r : b_\mathcal{A} \in \mathcal{C}_{h}^{r}\}  \}$ where $\mathcal{C}_h^r$ the chain  that the honest player $h$ has at round $r$.
\end{theorem}

The verifier outputs $\neg Q(\mathcal{C}_B)$. Thus it is necessary that $\pi_\mathcal{A} {\geq}_m \pi_B$. We show that $\pi_\mathcal{A} {\geq}_m \pi_B$ is a negligible event.

Let the levels of comparison decided by the verifier be $\mu_\mathcal{A}$ and $\mu_B$ respectively. Let $b_0 = LCA(\pi_\mathcal{A}, \pi_B)$. Let $\mu'_B$ be the adequate level of proof $\pi_B$  with respect to block $b_0$. Call $\alpha_\mathcal{A} = \pi_\mathcal{A} \upchain^{\mu_\mathcal{A}}\{b_0:\}$, $\alpha'_B = \pi_B \upchain^{\mu'_B}\{b_0:\}$.

From Corollary \ref{cor:adversarial_proof_scheme} we have that the adversarial proof consists of a smooth interlink subchain followed by a thorny interlink subchain. We will refer to the smooth part of $\alpha_\mathcal{A}$ as $\alpha^{\mathcal{S}}_\mathcal{A}$ and to the thorny part as $\alpha^{\mathcal{T}}_\mathcal{A}$.

Our proof construction is based on the following intuition: we consider that $\alpha_\mathcal{A}$ consists of three distinct parts $\alpha_\mathcal{A}^1, \alpha_\mathcal{A}^2, \alpha_\mathcal{A}^3$ with the following properties.

Consider $b_0 = LCA(\pi_\mathcal{A}, \pi_B)$ the fork point between $\pi_\mathcal{A} \upchain^{\mu_\mathcal{A}}, \pi_B \upchain^{\mu_B}$ and $b_1 = LCA (\alpha^{\mathcal{S}}_\mathcal{A}, \mathcal{C}_B)$ the fork point between $\pi^{\mathcal{S}}_\mathcal{A} \upchain^{\mu_\mathcal{A}}, \mathcal{C}_B $ at the zero level as the honest prover could observe. Part
$\alpha_\mathcal{A}^1$ contains the blocks between $b_0$ exclusive and $b_1$ inclusive generated during the set of consecutive rounds $\mathcal{S}_1$ and $\vert  \alpha_\mathcal{A}^1 \vert = k_1$.
Consider $b_2$ the last block in $\alpha_\mathcal{A}$ generated by an honest player. Part $\alpha_{\mathcal{A}}^2$ contains the blocks between $b_1$ exclusive and $b_2$ inclusive generated during the set of consecutive rounds $\mathcal{S}_2$ and $\vert  \alpha_\mathcal{A}^2 \vert = k_2$. Consider $b_3$ the next block of $b_2$ in $\alpha_\mathcal{A}$. Then $\alpha_{\mathcal{A}}^3 = \alpha_\mathcal{A}[b_3:]$ and $\vert  \alpha_\mathcal{A}^3 \vert = k_3$ of all adversarially generated blocks generated during the set of rounds $\mathcal{S}_3$. So, $\vert \alpha_\mathcal{A} \vert = k_1 + k_2 + k_3$ and we will show that $\vert \alpha_\mathcal{A} \vert < \vert \alpha_B \vert$ .

The above are illustrated, among other, in Parts I, II of Figure \ref{fig:proof_velvet}.

\begin{figure}[h!]
	\begin{center}
    \includegraphics[scale=0.65]{figures/proof_velvet-crop.pdf}
	\end{center}
	\caption{\textit{ Wavy lines imply one or more blocks. Dashed lines and arrows imply interlink pointers to superblocks. \textbf{I}: the three round sets in two competing proofs at different levels, \textbf{II}: the corresponding 0-level blocks implied by the two proofs, \textbf{III}: blocks participating in chain $\mathcal{C}_B$ and block set $\widetilde{\mathcal{C}}_\mathcal{A}$ from the verifier's perspective.}}
    \label{fig:proof_velvet}
\end{figure}

We will now show three successive claims under velvet fork conditions: First that $\alpha_\mathcal{A}^1$ contains only a few blocks. Second,  $\alpha_\mathcal{A}^2$ contains only a few blocks. And third, the adversary is able to produce a winning $a_\mathcal{A}$ with negligible probability.\\

\textbf{Claim 1:} $\alpha_\mathcal{A}^1 = \alpha_\mathcal{A}(b_0 : b_1]$ contains only a few blocks. We have defined $b_0 = LCA(\pi_\mathcal{A}, \pi_B)$ and $b_1 = LCA(\alpha^{\mathcal{S}}_\mathcal{A}, \alpha'_B \downarrow)$. First observe that there are no thorny blocks in $\alpha_\mathcal{A}^1$ since $\alpha_\mathcal{A}^1[-1] = b_1$ is a smooth block. This means that if $b_1$ was generated at round $r_{b_1}$ and $\alpha^{\mathcal{S}}_\mathcal{A}[-1]$ in round $r$, then $r \geq r_{b_1}$. So, $\alpha_\mathcal{A}^1$ consists a valid chain. We show the statement considering the two possible cases for the relation of $\mu_\mathcal{A}, \mu'_B$.

\textit{\underline{Claim 1a}:} If $\mu'_B \leq \mu_A$ then they are completely disjoint. In such a case of inequality, every block in $\alpha_A$ would also be of lower level $\mu'_B$. Because of the adequate level $\mu'_B$ we know that $C\{b:\}\upchain^{\mu'_B} = \pi\{b:\}\upchain^{\mu'_B}$\cite{nipopows}. Subsequently, any block in $\pi_A\upchain^{\mu_A}\{b:\}[1:]$ would also be included in proof $\alpha'_B$, but $b=LCA(\pi_A, \pi_B)$ so there can be no succeeding block common in $\alpha_A, \alpha'_B$.

\textit{\underline{Claim 1b}:} If  $\mu'_B > \mu_A$ then $\vert \alpha_A[1:]  \cap \alpha'_B\downarrow[1:] \vert = k_1 \leq g(2^{\mu'_B - \mu_A})$.\\
Let's call $b$ the first block in $\alpha'_B$ after block $b_0$. Suppose for contradiction that $k_1 > g(2^{\mu'_B - \mu_A})$.  Since block $b$ of level $\mu'_B$ is also of level $\mu_A$, the adversary could include it in the proof but $b$ cannot exist in both $\alpha_A, \alpha'_B$ since $\alpha_A \cap \alpha'_B = \emptyset$ by definition. In case that the adversary chooses not to include $b$ in the proof then she can include no other blocks of $\mathcal{C}_B$ in her proof, since it would not consist a valid chain. Therefore, the adversary can include at most the $\mu_\mathcal{A}$ upgraded  blocks  between  $b_0$, $b$, which are expected to be equal to $g(2^{\mu'_B - \mu_A})$.

We conclude that $\vert \alpha^{\mathcal{S}}_\mathcal{A} \cap \alpha'_B\downarrow[1:] \vert = k_{1} \leq g(2^{\mu'_B - \mu_\mathcal{A}}) $, where $g$ the velvet parameter denoting the percentage of upgraded honest parties.

Consequently, there are at least $\vert\alpha_\mathcal{A}\vert - k_1$ blocks after block $b$ in $\alpha_\mathcal{A}$ which are not honestly generated blocks existing in $\mathcal{C}_B$. In other words, there are $\vert \alpha_\mathcal{A} \vert - k_1$ blocks after block $b$ in $\alpha_\mathcal{A}$, which are either thorny blocks existing in $\mathcal{C}_B$ either don't belong in $\mathcal{C}_B$.\\

\textbf{Claim 2.}
Part $\alpha_\mathcal{A}^2 = \alpha_\mathcal{A}(b_1:b_2]$ consists of only a few blocks. Let $ \vert \alpha_\mathcal{A}^2 \vert = k_2$. We have defined $b_2 = \alpha_\mathcal{A}^2[-1]$ to be the last block generated by an honest player in $\alpha_\mathcal{A}$. Consequently no thorny block exists in $\alpha_\mathcal{A}^2$, so all blocks in this part belong in a proper zero-level chain $\mathcal{C}_\mathcal{A}^2$.  Let $r_{b_1}$ be the round at which $b_1$ was generated. Since $b_1$ is the last block in $\alpha_\mathcal{A}$ which belongs in $\mathcal{C}_B$, then $\mathcal{C}_\mathcal{A}^2$ is a fork chain to $\mathcal{C}_B$ at some block $b'$ generated at round $r' \geq r_{b_1}$.

Let $r_2$ be the round when $b_2$ was generated by an honest party. Because an honest party has chain $\mathcal{C}_B$ at later round $r_3$ when the proof $\pi_B$ is constructed and because of the Common Prefix property on parameter $k_{2\downarrow} = g \cdot 2^{\mu_\mathcal{A}} \cdot k_2$, we conclude that $k_2 \leq k$, where $k$ is the Common Prefix parameter as defined in the Backbone series of papers\cite{backbone}.
% XXX: move definition of $k$ into preliminaries section

\textbf{Claim 3.} The adversary may submit a suffix proof such that $\alpha_\mathcal{A} \geq \alpha_B$ with negligible probability.
As argued earlier the last $k_3$ blocks included in $\alpha_\mathcal{A}$ are all thorny blocks. In the worst case  all $k_3 $ blocks are sewed from $\mathcal{C}_B$. This is the worst case scenario since each adversarially generated block in $\mathcal{C}_B$ may have dropped one smooth block out of the chain because of selfish mining. Considering this scenario, because of the strengthened Honest Majority Assumption for $(1/3)$-bounded adversary, Theorem 3 for Chain Quality guarantees that the majority of the blocks in $\mathcal{C}_B$ was computed by honest parties, thus the honestly generated blocks in $\mathcal{C}_B$ for the same round set sum to more amount of hashing power.\\
From all the above Claims we have that:\\
In the first round set, because of the common underlying chain:
\begin{equation} \label{eq_v_round_set_1}
2^{\mu_\mathcal{A}} \vert \alpha_\mathcal{A}^{k_1} \vert \leq 2^{\mu'_B} \vert \alpha'{_B^{k_1}} \vert
\end{equation}
Because of the adoption by an honest party of chain $\mathcal{C}_B$ at a later round $r_3$, we have for the second round set:
\begin{equation} \label{eq_v_round_set_2}
	2^{\mu_\mathcal{A}} \vert \alpha_\mathcal{A}^{k_2} \vert \leq 2^{\mu'_B} \vert \alpha'{_B^{k_2}} \vert
\end{equation}
In the third round set, because of good Chain Quality under the strengthened Honest Majority Assumption and Theorem 3 we have:
\begin{equation} \label{eq_v_round_set_3}
	2^{\mu_\mathcal{A}} \vert \alpha_\mathcal{A}^{k_3} \vert < 2^{\mu'_B} \vert \alpha'{_B^{k_3}} \vert
\end{equation}
Consequently we have:

\begin{equation*}
	2^{\mu_\mathcal{A}} ( \vert \alpha_\mathcal{A}^{k_1} \vert + \vert \alpha_\mathcal{A}^{k_2} \vert + \vert \alpha_\mathcal{A}^{k_3} \vert ) < 2^{\mu'_B} ( \vert \alpha'{_B^{k_1}} \vert + \vert \alpha'{_B^{k_2}} \vert + \vert \alpha'{_B^{k_3}} \vert) \Rightarrow
\end{equation*}

\begin{equation} \label{eq_v_all_round_sets}
	2^{\mu_\mathcal{A}} \vert \alpha_\mathcal{A} \vert < 2^{\mu'_B} \vert \alpha'{_B} \vert
\end{equation}

Therefore we have proven that $2^{\mu'_B} \vert \pi_B \upchain^{\mu'_B} \vert > 2^{\mu_\mathcal{A}} \vert \pi_\mathcal{A}^{\mu_\mathcal{A}} \vert$. From the definition of $\mu_B$, we know that $2^{\mu_B} \vert \pi_B \upchain^{\mu_B} \vert > 2^{\mu'_B} \vert \pi_B \upchain^{\mu'_B} \vert$ because it was chosen $\mu_B$ as level of comparison by the Verifier. So we conclude that $2^{\mu_B} \vert \pi_B \upchain^{\mu_B} \vert > 2^{\mu_\mathcal{A}} \vert \pi_\mathcal{A} \upchain^{\mu_\mathcal{A}} \vert$.

\begin{flushright}
	$\square$
\end{flushright}

\section{Infix Proofs}\label{sec:infix}
NIPoPoW infix proofs answer any predicate which depends on blocks appearing anywhere in the chain, except for the $k$ suffix for stability reasons. For example, consider the case where a client has received a transaction inclusion proof for a block $b$ and requests an infix proof so as to verify that $b$ is included in the current chain.
Because of the described protocol update for secure NIPoPoW suffix proofs, the infix proofs construction has to be altered as well. In order to construct secure infix proofs under velvet fork conditions, we suggest the following additional protocol patch: each upgraded miner constructs and updates an authenticated data structure for all the blocks in the chain. We suggest Merkle Mountain Ranges \emph{(MMR)} for this structure. Now a velvet block's header additionally includes the root of this MMR.

After this additional protocol change the notion of a smooth block changes as well. Smooth blocks are now considered the blocks that contain truthful interlinks and valid MMR root too. A valid MMR root denotes the MMR that contains all the blocks in the chain of an honest full node. Note that a valid MMR contains all the blocks of the longest valid chain, meaning both smooth and thorny. An invalid MMR constructed by the adversary may contain a block of a fork chain. Consequently an upgraded prover has to maintain a local copy of this MMR locally, in order to construct correct proofs. This is crucial for the security of infix proofs, since keeping the notion of a smooth block as before would allow an adversary to produce a block $b$ in an honest party's chain, with $b$ containing a smooth interlink but invalid MMR, so she could succeed in providing an infix proof about a block of a fork chain.

Considering this addtional patch we can now define the final algorithms for the honest miner, infix and suffix prover, as well as for the infix verifier. Because of the new notion of smooth block, the function \textit{isSmoothBlock()} of Algorithm \ref{alg:smooth_chain_suffix} needs to be updated, so that the validity of the included MMR root is also checked. The updated function is given in Algorithm \ref{alg:isSmoothBlock_infix}. Considering that input $\chain_S$ is computed using Algorithm \ref{alg:smooth_chain_suffix} with the updated \textit{isSmoothBlock'()} function, \emph{Velvet updateInterlink} and \emph{Velvet Suffix Prover} algorithms remain the same as described in Algorithms \ref{alg:updateInterlink}, \ref{alg:velvet_suffix_prover} repsectively. The velvet infix prover and infix verifier algorithms are given in Algorithms \ref{alg:velvet_infix_prover}, \ref{alg:velvet_infix_verifier} respectively. Details about the construction and verification of an MMR and the respective inclusion proofs can be found in \cite{ct}.
Note that equivalent solution could be formed by using any authenticated data structure that provides inclusion proofs of size logarithmic to the length of the chain. We suggest MMRs because of they come with efficient update operations.

\begin{algorithm}[h]
	\caption{\label{alg:isSmoothBlock_infix}Function isSmoothBlock'() for infix proof support}
	\begin{algorithmic}[1]
			\Function{\sf isSmoothBlock'}{$B$}
				\If{$B = \mathcal{G}$}
					\State\Return{$\true$}
				\EndIf
				\For{$p \in B.\textsf{interlink}$}
						\If{$\lnot \textsf{isSmoothPointer}(B, p)$}
							\State\Return{$\false$}
						\EndIf
				\EndFor
				\State\Return{$\textsf{containsValidMMR}(B)$}
			\EndFunction
	\end{algorithmic}
\end{algorithm}

\begin{algorithm}[h]
	\caption{\label{alg:velvet_infix_prover}Velvet Infix Prover}
	\begin{algorithmic}[1]
			\Function{\sf ProveInfixVelvet}{$\chain_S, b$}
				\Let{(\pi,\chi)}{\textsf{ProveVelvet}(\chain_S)}
				\Let{\textsf{tip}}{\pi[-1]}
				\Let{\pi_b}{\textsf{MMRinclusionProof}(tip, b)}
				\State\Return{$(\pi_b,(\pi, \chi))$}
			\EndFunction
	\end{algorithmic}
\end{algorithm}

\begin{algorithm}[h]
	\caption{\label{alg:velvet_infix_verifier}Velvet Infix Verifier}
	\begin{algorithmic}[1]
			\Function{\sf VerifyInfixVelvet}{$b, (\pi_b,(\pi,\chi))$}
				\Let{\textsf{tip}}{\pi[-1]}
				\State\Return{\textsf{VerifyInclProof(tip.$root_{MMR}$, $\pi_b$, $b$)}}
			\EndFunction
	\end{algorithmic}
\end{algorithm}


\newpage
\appendix
\section{Chainsewing Attack Simulation}
\label{sec:simulation}

To measure the success rate of the chainsewing attack against the na\"ive NIPoPoW construction described in Section~\ref{sec:attack}, we implemented a simulation to estimate the probability of the adversary generating a winning NIPoPoW against the honest party. Our experimental setting is as follows. We fix $\mu_A = 0$ and $\mu_B = 10$ as well as the required length of the suffix $k = 15$. We fix the adversarial mining power to $t = 1$ and $n = 5$ which gives a $20\%$ adversary. We then vary the NIPoPoW security parameter for the $\pi$ portion from $m = 3$ to $m = 30$. We then run $100$ Monte Carlo simulations and measure whether the adversary was successful in generating a competing NIPoPoW which compares favourably against the adversarial NIPoPoW.

For performance reasons, our model for the simulation slightly deviates from the Backbone model on which the theoretical analysis of Section~\ref{sec:analysis} is based and instead follows the simpler model of Ren~\cite{nakamoto-simple}. This model favours the honest parties, and so provides a lower bound for probability of adversarial success, strengthening our results. Here, block arrival is modelled as a Poisson process and blocks are deemed to belong to the adversary with probability $t / n$, while they are deemed to belong to the honest parties with probability $(n - t) / n$. Block propagation is assumed instant and every party learns about a block as soon as it is mined. As such, the honest parties are assumed to work on one common chain and the problem of non-uniquely successful rounds does not occur.

\begin{figure}[h!]
	\begin{center}
		\iftwocolumn
			\includegraphics[width=\columnwidth]{figures/attack-confidence.pdf}
		\else
			\includegraphics[width=0.7 \columnwidth]{figures/attack-confidence.pdf}
		\fi
	\end{center}
	\caption{The measured probability of success of the Chainsewing attack mounted under our parameters for varying values of the security parameter $m$. Confidence intervals at $95\%$.}
	\label{fig:confidence}
\end{figure}

We consistently find a success rate of approximately $0.26$ which remains more or less constant independent of the security parameter, as expected. We plot our results with $95\%$ confidence intervals in Figure~\ref{fig:confidence}. This is in contrast with the best previously known attack in which, for all examined values of the security parameter, the probability of success remains below $1\%$.


%\section{Location}
%Note that in the new ACM style, the Appendices come before the References.

\begin{acks}
% TODO: For the submission, don't include acknowledgments since they would most likely deanonymize you.
\end{acks}


\bibliographystyle{ACM-Reference-Format}
\bibliography{refs}

\end{document}
