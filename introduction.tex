\section{Introduction}
Blockchain systems such as Bitcoin~\cite{nakamoto} and
Ethereum~\cite{buterin,wood} have a predetermined expected rate of block
production and maintain chains of blocks that are growing linearly
with time. A node synchronizing with the rest of the blockchain
network for the first time therefore has to download and validate the whole
chain, if it does not wish to rely on a trusted third party~\cite{taxonomy}. While a lightweight
node (SPV) can avoid downloading and validating transactions beyond their
interest, it must still download the block headers that contain the
proof-of-work~\cite{pow} of each block in order to determine which chain contains
the most work. The block headers, while smaller by a significant constant
factor, still grow linearly with time. An Ethereum node synchronizing for the
first time must download more than $4$ GB of block header data for the purpose
of proof-of-work verification, even if it does not download any
transactions. This has become a central problem to the usability of blockchain
systems, especially for vendors who use mobile phones to accept payments
and sit behind limited internet bandwidth. They are forced to make a difficult
choice between decentralization and the ability to start accepting payments in a
timely manner.

Towards the goal of alleviating the burden of this download for SPV clients, a
number of \emph{superlight} clients has emerged.
%These clients are able to
%choose the best proof-of-work chain by only requesting a small number of
%\emph{sample} block headers instead of all the block headers.
These protocols give rise to  Non-Interactive Proofs of Proof-of-Work (NIPoPoW)
~\cite{nipopows}, which are short strings
 that ``compress'' the proof-of-work information of the underlying chain by sending a selected sample of block headers.
%It has been shown
%that the block headers in these proofs are secure representatives of the
%proof-of-work of the underlying chain:
The necessary security property of such proofs is that a
minority adversary can only convince a
NIPoPoW client that a certain transaction is confirmed, only if they can
convince an SPV client, too.

There are two general directions for superlight client implementations: In the
\emph{superblock}~\cite{nipopows,compactsuperblocks} approach, the client
relies on \emph{superblocks}, blocks that have achieved much better
proof-of-work than required for block validity. In the
\emph{FlyClient}~\cite{flyclient} approach, blocks are randomly sampled and committed as in a $\Sigma$-protocol (e.g., Schnorr's discrete-log protocol~\cite{schnorr}) and then using
the Fiat--Shamir heuristic~\cite{fiatshamir} a non-interactive proof is calculated. The number of block headers
that need to be sent then grows only logarithmically with time. The NIPoPoW
client, which is the proof \emph{verifier} in this context, still relies on a connection to full nodes,
who, acting as \emph{provers},  perform the sampling of blocks from the full
blockchain.
No trust assumptions are made for these provers, as the
verifier can check the veracity of their claims. As long as the verifier is
connected to at least one honest prover (an assumption also made in the SPV
protocol~\cite{eclipse,eclipse-ethereum}), they can arrive at the correct claim.

In both approaches, the verifier must check that the blocks
sampled one way or another were generated in the same order as presented by the prover. As such, each block in the proof must contain a
pointer to the previous block in the proof. As blocks in these proofs are far
apart in the underlying blockchain, the legacy \emph{previous block pointer},
which typically appears within block headers, does not suffice.
Both approaches require modifications to the consensus layer of the underlying
blockchain to work. In the case of superblock NIPoPoWs, the block header must be
modified to include, in addition to a pointer to the previous block, pointers to
a small amount of recent high-proof-of-work blocks. In the case of FlyClient,
each block must additionally contain pointers to all previous blocks in the
chain. Both of these modifications can be made efficiently by organizing these
pointers into Merkle Trees~\cite{merkle} or Merkle Mountain Ranges~\cite{ct,mmr}
whose root is stored in the block header. The inclusion of extra pointers within
blocks is termed \emph{interlinking the chain}~\cite{popow}.

The modified block format, which includes the extra pointers, must be respected
and validated by full nodes and thus requires a hard or soft fork. However, even soft forks require the approval of a supermajority of
miners, and new features that are considered non-essential by the community have
taken years to receive approval~\cite{segwit}. Towards the goal of implementing
superlight clients sooner, we study the question of whether it is possible to
deploy superlight clients without a soft fork. We propose a series of
modifications to blocks that are \emph{helpful but untrusted}. These
modifications mandate that some extra data is included in each block. The extra
data is placed inside the block by upgraded miners only, while the rest of the
network does not include the additional data into the blocks and does not verify
its inclusion, treating them merely as comments. To maintain backwards
compatibility, contrary to a soft fork, upgraded miners must accept blocks that
do not contain this extra data that have been produced by unupgraded miners, or
even blocks that contain invalid or malicious such extra data produced by a
mining adversary. This acceptance is necessary in order to avoid causing a chain
split with the unupgraded part of the network. Such a modification to the
consensus layer is termed a \emph{velvet fork}~\cite{velvet}.
A summary of our contributions in this paper is as follows:
\begin{enumerate}
  \item We illustrate that, contrary to claims of previous work, superlight
        clients designed to work in a soft fork cannot be readily plugged into a velvet fork and expected to work. We present a novel and insidious
        attack termed the \emph{chain-sewing} attack which thwarts the defenses of previous proposals and allows even a minority adversary to cause
        catastrophic failures.
  \item We propose the first \emph{backwards-compatible superlight client}. We put forth an interlinking mechanism implementable through a velvet fork. We then construct a superblock NIPoPoW protocol on top of the velvet forked chain and show it allows to build superlight clients for various statements regarding the blockchain state via both ``suffix'' and ``infix'' proofs.
  \item We prove our construction secure in the synchronous static difficulty model against adversaries bounded to $1/3$ of the mining power of the honest upgraded nodes. As such, our protocol works even if a constant minority of miners adopts it.
\end{enumerate}

\noindent
\textbf{Previous work.} Proofs of Proof-of-Work have been proposed in the
context of superlight clients~\cite{nipopows,flyclient},
cross-chain communication~\cite{pow-sidechains,burn,crosschain-sok}, as well as
local data consumption by smart contracts~\cite{derivatives}. Chain interlinking
has been deployed in production both since genesis~\cite{ergo,nimiq} and using
hard forks~\cite{heartwood-flyclient}, and relevant verifiers have been
implemented~\cite{gglou,nipopow-gas}. They have
been conjectured to work in velvet fork conditions~\cite{nipopows} (we show
here that these conjectures are ill-informed in the light of our chain-sewing
attack). Velvet forks~\cite{velvet} have been studied for a variety of other
applications and have been deployed in practice~\cite{gtklocker}. In this work,
we focus on consensus state compression. Such compression has been explored in
the hard-fork setting using zk-SNARKS~\cite{coda} as well as in the
Proof-of-Stake setting~\cite{pos-sidechains}. Complementary to consensus state
compression (i.e., the compression of block headers and their ancestry) is
compression of application state, namely the State Trie, the UTXO, or
transaction history. There is a series of works complementary and composable with ours that
discusses the compression of application state~\cite{edrax,ethanos}.

\noindent
\textbf{Organization.}
The structure of the paper is as follows. In Section~\ref{sec:preliminaries}, we describe the typical superblock NIPoPoW protocol as designed for a soft fork (for a detailed description refer to Appendix~\ref{sec:nipopows_protocol} and \cite{nipopows}).
%give a brief overview of the \emph{backbone model} and its notation; as we
%heavily leverage the machinery of the model, the section is a necessary
%prerequisite to follow the analysis. The rest of the paper is structured in the
%form of \emph{a proof and refutation}~\cite{lakatos}. We believe this form is
%more digestible.
In Section~\ref{sec:velvet}, we discuss the velvet model and
some initial definitions, and we present a first attempt towards a velvet
NIPoPoW scheme which was presented in previous work. 
%We discuss the informal argument of why it seems to be secure, which we \emph{refute} with 
A concrete attack is explored
in Section~\ref{sec:attack} (we give simulation results and concrete parameters
for our attack in Appendix~\ref{sec:simulation}). In Section~\ref{sec:construction}, we patch the
scheme and put forth our more elaborate and novel Velvet NIPoPoW construction.
We analyze it and formally prove it secure in Appendix~\ref{sec:analysis}. Our scheme at this
point allows verifiers to decide which blocks form a \emph{suffix} of the
longest blockchain and thus the protocol supports \emph{suffix proofs}. We
extend our scheme to allow any block of interest within the blockchain to be
demonstrated to a prover in a straightforward manner in
Section~\ref{sec:infix}, giving a full \emph{infix proof} protocol. The latter
protocol can be used in practice and can be deployed today in real blockchains,
including Bitcoin, to confirm payments achieving both decentralization and
timeliness, solving a major outstanding dilemma in contemporary blockchain
systems.
