\begin{thm}{\textbf{Suffix Proofs Security under velvet fork}}
	Assuming honest majority under velvet fork conditions (\ref{defn:velvet_honest_majority}) such that $t \leq (1 - \delta) \dfrac{n_h}{2}$ where $n_h$ the number of upgraded honest players, the non-interactive proofs-of-proof-of-work construction for computable k-stable monotonic suffix-sensitive predicates under velvet fork conditions in a typical execution is secure.
\end{thm}
\textit{Proof.} By contradiction. Let $Q$ be a k-stable monotonic suffix-sensitive chain
predicate. Assume for contradiction that NIPoPoWs under velvet fork on $Q$ is insecure. Then, during an execution at some round  $r_3$, $Q(\mathcal{C})$ is defined and the verifier $V$ disagrees with some honest participant. $V$ communicates with adversary $\mathcal{A}$ and honest prover $B$. The verifier receives proofs $\pi_\mathcal{A}, \pi_B$ which are of valid structure. Because $B$ is honest, $\pi_B$ is a proof constructed
based on underlying blockchain $\mathcal{C}_B$ (with $\pi_B \subseteq \mathcal{C}_B$), which $B$
has adopted during round $r_3$ at which $\pi_B$ was generated. Consider
$\widetilde{\mathcal{C}}_\mathcal{A}$ the set of blocks defined as
$\widetilde{C}_\mathcal{A} = \pi_\mathcal{A} \cup \{ \bigcup \{\mathcal{C}_h^r\{:b_\mathcal{A}\}:  \forall b_\mathcal{A} \in \pi_\mathcal{A}, \exists h,r : b_\mathcal{A} \in 
\mathcal{C}_{h}^{r}\}  \}$ where $\mathcal{C}_h^r$ the chain 
 that the honest player $h$ has at round $r$.

The verifier outputs $\neg Q(\mathcal{C}_B)$. Thus it is necessary that $\pi_\mathcal{A} {\geq}_m \pi_B$.
We show that $\pi_\mathcal{A} {\geq}_m \pi_B$ is a negligible event.

Let the levels of comparison decided by the verifier
be $\mu_\mathcal{A}$ and $\mu_B$ respectively. Let $b = LCA(\pi_\mathcal{A}, \pi_B)$. Let $\mu'_B$ be the adequate level of proof
$\pi_B$  with respect to block $b$. Call $\alpha_\mathcal{A} = \pi_\mathcal{A} \uparrow^{\mu_\mathcal{A}}\{b:\}$,
$\alpha'_B = \pi_B \uparrow^{\mu'_B}\{b:\}$.

From Corollary \ref{cor:adversarial_proof_scheme} we have that the adversarial proof 
consists of a smooth interlink subchain followed by a thorny interlink subchain. We will refer to the smooth part of $\alpha_\mathcal{A}$ as $\alpha^{\mathcal{S}}_\mathcal{A}$ and to the thorny part as $\alpha^{\mathcal{T}}_\mathcal{A}$.   

%%%%%%%%   reconsider this paragraph
Our proof construction is based on the following intuition: we show that
$\alpha_\mathcal{A}$ consists of three distinct parts $(\alpha_\mathcal{A}
^1, \alpha_\mathcal{A}^2, \alpha_\mathcal{A}^3)$, where $\vert 
\alpha_\mathcal{A}^1 \vert = k_1, \vert \alpha_\mathcal{A}^2 \vert = k_2$
and $\vert \alpha_\mathcal{A}^3 \vert = k_3$. So $\vert \alpha_\mathcal{A} \vert = k_1 + k_2 + k_3$. Consider $\mathcal{S}_1,
\mathcal{S}_2, \mathcal{S}_3$ the sets of consecutive rounds where the
blocks of the three parts of $\alpha_\mathcal{A}$ are, respectively, 
generated. Consider $b_1 = LCA(\pi_\mathcal{A}, \pi_B)$ and $b_2 = LCA
(\alpha^{\mathcal{S}}_\mathcal{A}, \mathcal{C}_B)$, then part 
$\alpha_\mathcal{A}^1$ contains the blocks between $b_1$ and $b_2$. 
Consider $b_3$ the first thorny block in $\alpha_\mathcal{A}$, so $b_3 = \alpha_\mathcal{A}^{\mathcal{T}}[0]$, then 
$\alpha_\mathcal{A}^3 = \alpha_\mathcal{A}\{b_3:\}$. The second part 
$\alpha_\mathcal{A}^2$ contains the rest of the blocks in $\alpha_\mathcal{A}$. 

The above are illustrated, among other, in Parts I, II of Figure \ref{fig:proof_velvet}.

\begin{figure}[h!]
	\begin{center}
    \includegraphics[scale=0.6]{figures/proof_velvet.pdf}
	\end{center}
	\caption{\textit{ Wavy lines imply one or more blocks. Dashed lines and arrows imply
	interlink pointers to superblocks. \textbf{I}: the three round sets in two competing
	proofs at different levels, \textbf{II}: the corresponding 0-level blocks implied by the two proofs,
	\textbf{III}: blocks participating in chain $\mathcal{C}_B$ and block set $\widetilde{\mathcal{C}}_\mathcal{A}$ from the verifier's perspective.}}	
    \label{fig:proof_velvet}
\end{figure}

We will now show three successive claims under velvet fork conditions: First,
$\alpha^{\mathcal{S}}_\mathcal{A}$ and $\alpha'_B \downarrow$ are mostly
disjoint. Second, $a_\mathcal{A}$ contains mostly adversarially generated blocks. And third,
the adversary is able to produce this $a_\mathcal{A}$ with negligible probability.\\

\textbf{Claim 1:} $\alpha^{\mathcal{S}}_\mathcal{A}$ and
$\alpha'_B \downarrow$ are mostly disjoint. First consider that there are no thorny blocks in the adversary's proof between $b_1$ and $b_2$. This means that if $b_2$ was generated at round $r_{b_2}$ and $\alpha^{\mathcal{S}}_\mathcal{A}[-1]$ in round $r$, then $r \geq r_{b_2}$. For this case we show the statement considering the two possible cases for the relation of $\mu_\mathcal{A}, \mu'_B$.

\textit{\underline{Claim 1a}:} If $\mu'_B \leq \mu_A$ then they are completely disjoint. In such a case of inequality, every block in $\alpha_A$ would also be
of lower level $\mu'_B$. Because of the adequate level $\mu'_B$ we know that $C\{b:\}\uparrow^{\mu'_B} = \pi\{b:\}\uparrow^{\mu'_B}$\cite{NIPoPoWs}. Subsequently, any block in $\pi_A\uparrow^{\mu_A}\{b:\}[1:]$ would also be included in proof $\alpha'_B$, but $b=LCA(\pi_A, \pi_B)$ so there can be no succeeding block common in $\alpha_A, \alpha'_B$. \\

\textit{\underline{Claim 1b}:} If  $\mu'_B > \mu_A$ then $\vert \alpha_A[1:]  \cap \alpha'_B\downarrow[1:] \vert = k_1 \leq g(2^{\mu'_B - \mu_A})$.\\
Let's call $b$ the first block in $\alpha'_B$ after block $b_1$.
Suppose for contradiction that $k_1 > g(2^{\mu'_B - \mu_A})$.  Since block $b$ of level $\mu'_B$ is also of level $\mu_A$, the adversary could include it in the proof but $b$ cannot exist in both $\alpha_A, \alpha'_B$ since $\alpha_A \cap \alpha'_B = \emptyset$ by definition. In case that the adversary chooses not to include $b$ in the proof then she can include no other blocks of $C_B$ in her proof, since it would not consist a valid chain. Therefore, the adversary can include at most the $\mu_\mathcal{A}$ upgraded  blocks  between  $b_1$, $b$, which are expected to be equal to $g(2^{\mu'_B - \mu_A})$\\

Now consider the case where the adversary includes a thorny block $b_t = \alpha^{\mathcal{T}}_{\mathcal{A}}[0]$ after $b$ and before $b_2$, thus the inequality still
holds and because of Lemma \ref{lemm:thorny_after_thorny} no more honestly generated blocks can be included
in $\alpha_\mathcal{A}$ after $b_t$ and we can immediately proceed to Claim 3 of this proof.

We conclude that 
$\vert \alpha^{\mathcal{S}}_\mathcal{A} \cap \alpha'_B\downarrow[1:] \vert = k_{1} \leq g(2^{\mu'_B - \mu_\mathcal{A}})
$, where $g$ the velvet parameter denoting the percentage of upgraded honest parties. 

We conclude that there are at least $\vert\alpha_\mathcal{A}\vert - k_1$
blocks after block $b$ in $\alpha_\mathcal{A}$ which are not honestly generated blocks existing in
$\mathcal{C}_B$. In other words, there are $\vert \alpha_\mathcal{A} \vert - k_1$
blocks after block $b$ in $\alpha_\mathcal{A}$, which are either thorny blocks existing in $\mathcal{C}_B$ either don't belong in $\mathcal{C}_B$.\\

\textbf{Claim 2.} 
At least $k_3$ superblocks of $\alpha_\mathcal{A}$ are adversarially generated.
Consider that the block following $b_2$ in $\alpha_\mathcal{A}$ is a smooth one. Let's call this block $b'_2$. This means that $b'_2$ does not belong to $\mathcal{C}_B$ but to a fork chain. In this case round set $\mathcal{S}_2$ refers to the consecutive rounds from block $b_2$ and until the Common Prefix is established at $\mathcal{C}_B$ for that fork point. Consider $k_{2\downarrow}$ blocks in $\mathcal{C}_B$ are generated during $\mathcal{S}_2$. Then, bacause of the Common Prefix property on parameter $k_{2\downarrow}$, $\alpha_\mathcal{A}[k_{1}+k_{2}:]$ could contain no honestly generated blocks, so $k_2 \leq k$.

In the case that $b'_2$ is a thorny block, then because of Corollary \ref{cor:adversarial_proof_scheme} no more smooth block are included in $\alpha_\mathcal{A}$. So we consider that $k_2 = 0$ and can proceed to part $\alpha_\mathcal{A}^3$ and to Claim 3 of this proof.\\

\textbf{Claim 3.} The adversary may submit a suffix proof such that $\alpha_\mathcal{A} \geq \alpha_B$
with negligible probability.
%%% @TODO: make formal arguments
As argued earlier the last $k_3$ blocks included in $\alpha_\mathcal{A}$ are all thorny blocks. In the worst case  all $k_3 $ blocks
are sewed from $\mathcal{C}_B$. This is the worst case scenario since each adversarially
generated block in $\mathcal{C}_B$ may have dropped one smooth block out of the chain
because of selfish mining. Considering this scenario, because of the strengthened
Honest Majority Assumption for $(1/3)$-bounded adversary, Theorem 3 for Chain
Quality guarantees that the majority of the blocks in $\mathcal{C}_B$ was computed by
honest parties, thus the honestly generated blocks in $\mathcal{C}_B$ for the same round
set sum to more amount of hashing power.\\
From all the above Claims we have that:\\
In the first round set, because of the common underlying chain:
\begin{equation} \label{eq_v_round_set_1}
2^{\mu_\mathcal{A}} \vert \alpha_\mathcal{A}^{k_1} \vert \leq 2^{\mu'_B} \vert \alpha'{_B^{k_1}} \vert
\end{equation}
Because of the adoption by an honest party of chain $\mathcal{C}_B$ at a later round $r_3$, we
have for the second round set:
\begin{equation} \label{eq_v_round_set_2}
2^{\mu_\mathcal{A}} \vert \alpha_\mathcal{A}^{k_2} \vert \leq 2^{\mu'_B} \vert \alpha'{_B^{k_2}} \vert
\end{equation}
In the third round set, because of good Chain Quality under the strengthened Honest
Majority Assumption and Theorem 3 we have:
\begin{equation} \label{eq_v_round_set_3}
2^{\mu_\mathcal{A}} \vert \alpha_\mathcal{A}^{k_3} \vert < 2^{\mu'_B} \vert \alpha'{_B^{k_3}} \vert
\end{equation}
Consequently we have:

\begin{equation*}
2^{\mu_\mathcal{A}} ( \vert \alpha_\mathcal{A}^{k_1} \vert + \vert \alpha_\mathcal{A}^{k_2} \vert + \vert
\alpha_\mathcal{A}^{k_3} \vert ) < 2^{\mu'_B} ( \vert \alpha'{_B^{k_1}} \vert + \vert
\alpha'{_B^{k_2}} \vert + \vert \alpha'{_B^{k_3}} \vert) \Rightarrow
\end{equation*}

\begin{equation} \label{eq_v_all_round_sets}
2^{\mu_\mathcal{A}} \vert \alpha_\mathcal{A} \vert < 2^{\mu'_B} \vert \alpha'{_B} \vert
\end{equation}

Therefore we have proven that $2^{\mu'_B} \vert \pi_B \uparrow^{\mu'_B} \vert >
2^{\mu_\mathcal{A}} \vert \pi_\mathcal{A}^{\mu_\mathcal{A}} \vert$. From the definition of $\mu_B$, we know
that $2^{\mu_B} \vert \pi_B \uparrow^{\mu_B} \vert > 2^{\mu'_B} \vert \pi_B
\uparrow^{\mu'_B} \vert$ because it was chosen $\mu_B$ as level of comparison
by the Verifier. So we conclude that $2^{\mu_B} \vert \pi_B \uparrow^{\mu_B}
\vert > 2^{\mu_\mathcal{A}} \vert \pi_\mathcal{A} \uparrow^{\mu_\mathcal{A}} \vert$.

\begin{flushright}
$\square$
\end{flushright}