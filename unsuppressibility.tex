\section{$Q$-block Suppression}
\begin{definition}[block property, $Q$-block]
    \cite{dionyziz}
    A block property is a predicate $Q$ defined on a hash output $h \in \{ 0, 1 \}^\kappa$. Given  a block property $Q$, a valid block with hash h is called a $Q$-block if $Q(h)$ is true.
\end{definition}

\begin{lemma}[Unsuppressibility]\cite{dionyziz}
    Consider a collection of polynomially many block properties $\mathcal{Q}$. In a typical execution every set of consecutive rounds U has a subset S of uniquely successful rounds such that
    \begin{itemize}
        \item $\lvert S \rvert \geq Y(U) - 2Z(U) - 2 \lambda f (\dfrac{t}{n-t} \cdot \dfrac{1}{1-f} + \epsilon)$
        \item for any $Q \in \mathcal{Q}$, $Q$-blocks generated during S follow the distribution as in an unsuppressed chain
        \item after the last round in S the blocks corresponding to S belong to the chain of any honest party.
    \end{itemize}
\end{lemma}


\begin{lemma}
   Consider Algorithm \ref{alg:updateInterlink} under velvet fork with parameter $g$ and (1/4)-bounded velvet honest majority. Let $U$ be a set of consecutive rounds $r_1 \cdots r_2$ and $\chain$ the chain of an honest party at round $r_2$ of a typical execution. Let $\chain^S_U = \{b \in \chain: \text{b is smooth} \wedge \text{b was generated during U} \}$. Let $\mu, \mu' \in \mathbb{N}$. Suppose $\chain'$ a $\mu'$ superchain containing only adversarial blocks generated during $U$ and suppose $\lvert \chain' \rvert > k$. Then it holds that $ 2^{\mu'}\lvert \chain' \rvert < 2^{\mu}\lvert \chain^S_U\upchain^{\mu} \rvert $.
   \label{lem:claim3_lemma}
\end{lemma}
\begin{proof} From the Unsuppressibility Lemma we have that there is a set of uniquely successful rounds $S$, subset of U, such that $\lvert S \rvert \geq Y(U) - 2Z(U) - \delta'$, where $\delta' = 2 \lambda f (\dfrac{t}{n-t} \cdot \dfrac{1}{1-f} + \epsilon)$. We also know that $Q$-blocks of the property collection $\mathcal{Q} = \{Q(h) = true: h \in \{0,1\}^\kappa, \widetilde{\mu} \leq log(\lvert \chain \rvert), h \leq 2^{-\widetilde{\mu}}T \}$ that were generated during $S$ are distributed as in an unsuppressed chain. Therefore for the number of interlinked $\mu$-blocks that were generated during $S$ it holds that  $\lvert \chain^S_U\upchain^{\mu} \rvert \geq (1-\epsilon)g2^{-\mu} \lvert S \rvert$. For the total number of $\mu'$-blocks the adversary generated during $U$ it holds that $\lvert \chain' \rvert \leq (1+\epsilon)2^{-\mu'}Z(U)$. Then we have to show that $(1-\epsilon)g (Y(U) - 2Z(U) - \delta' ) > (1+\epsilon)Z(U)$ or $((1+\epsilon)+2g(1-\epsilon))Z(U) < g(1-\epsilon)(Y(U) + \delta')$. Let $\alpha_z = (1+\epsilon)+2g(1-\epsilon)$ then we have $\alpha_z Z(U) < g(1-\epsilon)(Y(U) + \delta')$.

Because of the typical exexution we know that $Y(U), Z(U)$ cannot deviate much from their expected values. In particular, we have that\cite{backbone}: $Z(U) < \mathbb{E}[Z(U)] + \epsilon \mathbb{E}[X(U)]$ and $Y(U) > (1-\epsilon)\mathbb{E}[Y(U)]$. From the Backbone analysis we also know that $\mathbb{E}[Z(U)] < (1+\dfrac{\delta}{2})f\dfrac{t}{n-t} \lvert U \rvert$, $\mathbb{E}[X(U)] < pq(n-t) \lvert U \rvert$ and $\mathbb{E}[Y(U)] > f(1-f) \lvert U \rvert$. By substituting the expectation values we have $ \alpha_z[(1+\dfrac{\delta}{2})f \dfrac{t}{n-t} \lvert U \rvert + \epsilon pq(n-t) \lvert U \rvert ] < (1- \epsilon)g[ (1-\epsilon)f(1-f) \lvert U \rvert - \delta' ] $ or $ \alpha_z f \lvert U \rvert \dfrac{t}{n-t}(1 + \dfrac{\delta}{2}) + \alpha_z \epsilon pq(n-t) \lvert U \rvert < (1- \epsilon)g[ (1-\epsilon)f(1-f) \lvert U \rvert - \delta' ] $ or \begin{equation*}
    \dfrac{t}{n-t} < \dfrac{ (1- \epsilon)g[ (1-\epsilon)f(1-f) \lvert U \rvert - \delta' ] - \alpha_z \epsilon pq(n-t) \lvert U \rvert }  { f \lvert U \rvert \alpha_z (1 + \dfrac{\delta}{2})}
\end{equation*} or \begin{equation*}
    \dfrac{t}{n-t} < \dfrac{  (1- \epsilon)g[ (1-\epsilon)f(1-f) - \dfrac{\delta'}{\lvert U \rvert} ] - \alpha_z \epsilon pq(n-t) }  { f \alpha_z (1 + \dfrac{\delta}{2})}
\end{equation*} But \begin{equation*}
    \dfrac{\epsilon pq(n-t)}{f(1+\dfrac{\delta}{2})} = \epsilon' \ll 1
\end{equation*} thus \begin{equation*}
    \dfrac{t}{n-t} < \dfrac{  (1- \epsilon)g[ (1-\epsilon)f(1-f) - \dfrac{\delta'}{\lvert U \rvert} ] }  { f \alpha_z (1 + \dfrac{\delta}{2})} - \epsilon'
\end{equation*} or \begin{equation*}
    \dfrac{t}{n-t} < \dfrac{g}{(1+\dfrac{\delta}{2})f\alpha_z} \cdot \dfrac{(1-\epsilon)^2 f(1-f) - \dfrac{(1-\epsilon)\delta'}{\lvert U \rvert} }{(1+\dfrac{\delta}{2})f\alpha_z} - \epsilon'
\end{equation*} Consequently we have that $\dfrac{t}{n-t} < \dfrac{1-\delta_v}{3}g$ which is the (1/4) velvet honest majority assumption, since \begin{equation*}
    \dfrac{1}{(1+\dfrac{\delta}{2})f\alpha_z} = \dfrac{1}{(1+\dfrac{\delta}{2})f(1+\epsilon) + (1+\dfrac{\delta}{2})f(1-\epsilon)2g} > \dfrac{1}{3}
\end{equation*} and \begin{equation*}
    \dfrac{(1-\epsilon)^2 f(1-f) - \dfrac{(1-\epsilon)\delta'}{\lvert U \rvert} }{(1+\dfrac{\delta}{2})f\alpha_z} > (1 - \delta_v)
\end{equation*}. The last holds because we have $\lvert \chain' \rvert > k$ so $\lvert U \rvert \gg (1-\epsilon)\delta'$.
\end{proof}