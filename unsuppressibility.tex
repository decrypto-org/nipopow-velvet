\section{$\mathcal{Q}$-block Suppression}
\begin{definition}[block property, $\mathcal{Q}$-block]
    \cite{dionyziz}
    A block property is a predicate $\mathcal{Q}$ defined on a hash output $h \in \{ 0, 1 \}^\kappa$. Given  a block property $\mathcal{Q}$, a valid block with hash h is called a $\mathcal{Q}$-block if $\mathcal{Q}(h)$ is true.
\end{definition}

\begin{lemma}[Unsuppressibility]\cite{dionyziz}
    In a typical execution every set of consecutive rounds U has a subset S of uniquely successful rounds such that
    \begin{itemize}
        \item $\lvert S \rvert \geq Y(U) - 2Z(U) - 2 \lambda f (\dfrac{t}{n-t} \cdot \dfrac{1}{1-f} + \epsilon)$
        \item $Q$-blocks generated during S follow the distribution expected in an unsuppressed chain
        \item after the last round in S the blocks corresponding to S belong to the chain of any honest party.
    \end{itemize}
\end{lemma}


\begin{lemma}
   Consider superblock NIPoPoWs under velvet fork with parameter $g$ and (1/4)-bounded velvet honest majority. Let $U$ be a set of consecutive rounds $r_1 ... r_2$ and $\chain_B$ the chain of an honest party at round $r_2$ of a typical execution. Let $\chain_B^U = \{b \in \chain_B: \text{b was generated during U} \}$. Let $\mu_\mathcal{A}, \mu_B \in \mathbb{N}$. Suppose $\chain'_\mathcal{A}$ a $\mu_\mathcal{A}$ superchain containing only adversarial blocks generated during $U$ and suppose $\vert \chain'_\mathcal{A} \vert > k$. Then it holds that $ 2^{\mu_\mathcal{A}}\lvert \chain'_\mathcal{A} \rvert < 2^{\mu_B}\lvert \chain_B^U\upchain^{\mu_B} \rvert $.
\end{lemma}
\begin{proof} From the Unsuppressibility Lemma we have that there is a set of uniquely successful rounds $S$, subset of U, such that $\lvert S \rvert \geq Y(U) - 2Z(U) - \delta'$, where $\delta' = 2 \lambda f (\dfrac{t}{n-t} \cdot \dfrac{1}{1-f} + \epsilon)$. We also know that $\mathcal{Q}$-blocks in $S$ are distributed as would be expected in an unsuppressed chain. Therefore the number of $\mu_B$-blocks in $S$ is $2^{-\mu_B} \lvert S \rvert$. Considering also the velvet parameter to calculate the number of interlinked $\mu_B$-blocks we have $\lvert \chain_B^U\upchain^{\mu_B} \rvert = g2^{-\mu_B} \lvert S \rvert$.  The total number of $\mu_\mathcal{A}$-blocks the adversary generated during $U$ is $2^{-\mu_\mathcal{A}}Z(U)$, therefore $\lvert \chain'_\mathcal{A} \rvert \leq 2^{-\mu_\mathcal{A}}Z(U)$. Then we have to show that $2^{\mu_B}\lvert \chain_B^U\upchain^{\mu_B} \rvert \geq g ( Y(U) - 2Z(U) - \delta' ) $ or $Z(U) < \dfrac{g}{3}(Y(U) - \delta')$.

But we know that\cite{backbone}: $\mathbb{E}[Z(U)] < (1+\dfrac{\delta}{2})f\dfrac{t}{n-t}$, $\mathbb{E}[X(U)] < pq(n-t)$ and $\mathbb{E}[Y(U)] > f(1-f)$. Additionaly, because of the typical execution we know that the random variable values cannot deviate much from their expected values. In particular we have\cite{backbone}: $Z(U) < \mathbb{E}[Z(U)] + \epsilon \mathbb{E}[X(U)]$ and $Y(U) > (1-\epsilon)\mathbb{E}[Y(U)]$. Then $\mathbb{E}[Z(U)] + \epsilon\mathbb{E}[X(U)] < \dfrac{g}{3}( (1-\epsilon)\mathbb{E}[Y(U)] -\delta')$ or $\mathbb{E}[Z(U)]  < \dfrac{g}{3}( (1-\epsilon)\mathbb{E}[Y(U)] -\delta') - \epsilon\mathbb{E}[X(U)]$ which concludes to $\dfrac{t}{n-t} < \dfrac{\dfrac{g}{3}((1-\epsilon)f(1-f)-\delta') - \epsilon pq(n-t)}{(1-\dfrac{\delta}{2})f}$. Consequently we have that $\dfrac{t}{n-t} < \dfrac{1-\delta_v}{3}g$ which is the (1/4) velvet honest majority assumption.
\end{proof}